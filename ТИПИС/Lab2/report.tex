\documentclass[a4paper,14pt]{extarticle}
\usepackage{../../tex-shared/preamble}

\renewcommand{\mylabnumber}{2}
\renewcommand{\mylabtitle}{Расчет числовых характеристик и 
                           энтропии дискретной случайной величины}
\renewcommand{\mysubject}{Теория информационных процессов и систем}
\renewcommand{\mylecturer}{Заикина Е.Н.}

\begin{document}
\begin{titlepage}
    
    \thispagestyle{empty}
    
    \begin{center}
        
        Министерство науки и Высшего образования Российской Федерации \\
        Севастопольский государственный университет \\
        Кафедра ИС
        
        \vfill

        Отчет \\
        по лабораторной работе №\mylabnumber \\
        \enquote{\mylabtitle} \\
        по дисциплине \\
        \enquote{\MakeTextUppercase{\mysubject}}

    \end{center}

    \vspace{1cm}

    \noindent\hspace{7.5cm} Выполнил студент группы ИС/б-17-2-о \\
    \null\hspace{7.5cm} Горбенко К. Н. \\
    \null\hspace{7.5cm} Проверил \\
    \null\hspace{7.5cm} \mylecturer

    \vfill

    \begin{center}
        Севастополь \\
        \the\year{}
    \end{center}

\end{titlepage}
\section{Цель работы}
\begin{itemize}
    \item Изучение способов описания дискретных случайных величин.
    \item Приобретение практических навыков расчета числовых характеристик
          и энтропии дискретной случайной величины по ее закону распределения.
\end{itemize}

\section{Ход работы}
\begin{enumerate}
    \item Получить у преподавателя вариант задания.
    \item Написать функцию, определяющую распределение вероятностей
          дискретной случайной величины в соответствии с заданным
          законом распределения.
    \item Проверить условие нормировки.
    \item Написать функцию для определения начального момента s-го порядка.
          Выписать соответствующую формулу.
    \item Найти начальный момент нулевого порядка. Объяснить результат.
    \item Написать функцию для определения математического ожидания.
          Выписать соответствующую формулу.
    \item Построить графики зависимости математического ожидания от параметров
          распределения.
    \item Написать функцию для определения центрального момента s-го порядка.
          Выписать соответствующую формулу
    \item Найти центральный момент нулевого порядка. Объяснить результат.
    \item Найти центральный момент первого порядка. Объяснить результат.
    \item Написать функцию для определения дисперсии. Выписать
          соответствующую формулу.
    \item Построить графики зависимости дисперсии от параметров распределения.
    \item Написать функцию для определения среднеквадратического отклонения.
          Выписать соответствующую формулу.
    \item Построить графики зависимости среднеквадратического отклонения
          от параметров распределения.
    \item Написать функцию для определения коэффициента асимметрии. Выписать
          соответствующую формулу.
    \item Построить графики зависимости коэффициента асимметрии от
          параметров распределения.
    \item Написать функцию для определения коэффициента эксцесса.
          Выписать соответствующую формулу.
    \item Построить графики зависимости коэффициента эксцесса от параметров
          распределения.
    \item Построить графики распределения вероятностей для разных параметров
          распределения.
    \item Написать функцию, определяющую интегральный закон распределения 
          дискретной случайной величины, подчиненной заданному закону 
          распределения.
    \item Построить графики интегрального закона распределения для
          разных параметров распределения
    \item Написать функцию для вычисления энтропии.
    \item Построить графики зависимости энтропии от параметров распределения.
    \item Сделать развернутые выводы по результатам исследований.
\end{enumerate}
\end{document}