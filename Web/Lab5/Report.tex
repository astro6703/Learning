\documentclass[a4paper,14pt]{extarticle}
\usepackage{../../tex-shared/report-layout}

\renewcommand{\mylabnumber}{5}
\renewcommand{\mylabtitle}{Исследование возможностей
ускорения разработки веб-приложений с использованием HAML}
\renewcommand{\mysubject}{Веб-технологии}
\renewcommand{\mylecturer}{Дрозин А.Ю.}

\begin{document}
\begin{titlepage}
    
    \thispagestyle{empty}
    
    \begin{center}
        
        Министерство науки и Высшего образования Российской Федерации \\
        Севастопольский государственный университет \\
        Кафедра ИС
        
        \vfill

        Отчет \\
        по лабораторной работе №\mylabnumber \\
        \enquote{\mylabtitle} \\
        по дисциплине \\
        \enquote{\MakeTextUppercase{\mysubject}}

    \end{center}

    \vspace{1cm}

    \noindent\hspace{7.5cm} Выполнил студент группы ИС/б-17-2-о \\
    \null\hspace{7.5cm} Горбенко К. Н. \\
    \null\hspace{7.5cm} Проверил \\
    \null\hspace{7.5cm} \mylecturer

    \vfill

    \begin{center}
        Севастополь \\
        \the\year{}
    \end{center}

\end{titlepage}

\section{Цель работы}
Изучить основные возможности языка HAML, приобрести практические
навыки реализации веб-страниц с использованием данной технологии.

\section{Задание на работу}

\subsection{Описание шаблона сайта}
\begin{enumerate}
    \item Необходимо с использованием языка HAML создать шаблон страниц
    пользовательской части сайта, содержащий общие блоки информации, такие
    как меню, ‘шапка’, ‘подвал’ и др. Шаблон разместить в отдельном файле.
    \item Реализовать страницы «Главная», «Обо мне», «Контакты», «Учеба»
    с использованием HAML, поместить результаты в отдельные файлы.
    \item Реализовать страницу «Фотоальбом» с использованием языка
    HAML. При реализации страницы воспользоваться циклическими
    операторами языка HAML для автоматизации процесса верстки фотоальбома.
    Информацию об изображениях хранить с использованием массива.
\end{enumerate}

\section{Ход работы}
Выделим со страниц общие части и вынесем их в отдельные файлы. На каждой
странице повторяются: навигационное меню, подвал, фоновая картинка,
включение общих стилей. Вынесем их в отдельные файлы в директорию \code{haml\textbackslash partials}

Содержимое файла \code{nav.haml}, описывающего навигационное меню сайта представлено далее:
\begin{lstlisting}
%nav
  %ul
    %li
      %span Web lab1
    %li
      %a{ href: "index.html" } Home
    %li
      %a{ href: "about.html" } About
    %li
      %a{ href: "interests.html" } My interests
    %li
      %a{ href: "learning.html"} Learning
    %li
      %a{ href: "photos.html" } Photos
    %li
      %a{ href: "contact.html" } Contact
    %li
      %a{ href: "test.html" } Test
    %li
      %a{ href: "history.html" } History
    %li
      %span{ id: "time" }
      %br
      %span{ id: "date" }
\end{lstlisting}

Содержимое файла \code{footer.haml}, описывающего подвал сайта представлено далее:
\begin{lstlisting}
%footer
  %section
    %header
      %h2 Футер
    %p Some useful links here
    %ul
      %li Link
      %li Another link
    %p
      And some info as well: Nunc eget viverra neque, non ullamcorper odio. Integer dolor sem, ultrices eu
      tempus dapibus, sagittis ac lacus. Duis rutrum turpis eu lectus imperdiet, ac
      sagittis orci faucibus. Vestibulum volutpat fringilla turpis vehicula lobortis.
      Interdum et malesuada fames ac ante ipsum primis in faucibus. Aliquam posuere
      tempor enim, et tempus tellus commodo mattis. Sed sollicitudin mauris vitae
      vestibulum malesuada.
\end{lstlisting}

Общее содержимое тега \code{head} сайта также вынесем в отдельный файл. Содержимое файла \code{head-shared.haml}
представлено далее.
\begin{lstlisting}
%meta{ name: "viewport", content: "width=device-width" }
%meta{ content: "no-cache", "http-equiv": "Cache-Control" }
%link{ href: "../src/pages/nav.css", "rel": "stylesheet", "type": "text/css" }
%link{ href: "../src/pages/shared.css", "rel": "stylesheet", "type": "text/css" }
\end{lstlisting}

Также вынесем фоновую картинку (файл \code{portrait.haml}):
\begin{lstlisting}
%img{ src: "../images/portrait.jpg", class: "portrait", alt: "Portrait picture in on the back. Don't mind id" }
\end{lstlisting}

\subsection{Формирование \code{.haml} документа страницы}
После вынесения общих частей в отдельные файлы необходимо найти способ включить
содержимое одного \code{.haml} файла в другой. Сделаем это с помощью следующей
команды:

\code{Haml::Engine.new(File.read(path)).render}

Такая команда должна вызываться каждый раз при включении содержимого одного файла в другой.
Упростим ее, вынеся в отдельный модуль (файл \code{shared\textbackslash globals.rb}):

\begin{lstlisting}
module Haml::Helpers
    def render(path)
        content = File.read(path)
        Haml::Engine.new(content).render
    end
end
\end{lstlisting}

Теперь для решения задачи необходимо просто вызвать функцию \code{render}.
Для включения этого файла в процесс парсинга \code{haml} в \code{html}
нужно при вызове команды его явно указывать:

\code{haml -r shared\textbackslash globals.rb file1.haml file2.html}

\subsection{Верстка страниц сайта}

\subsubsection{Домашняя страница}
Текст документа \code{index.haml} приведен далее:
\begin{lstlisting}
%html
  %head
      %title Homepage
      = render("./partials/head-shared.haml")
  %body
      = render("./partials/nav.haml")
      = render("./partials/portrait.haml")
      %div.main-content
      %header
          %h1 Портфолио
      %section
          %img.avatar{ src: "../images/avatar.jpg", alt: "author's avatar", title: "that's me!" }
          %header
          %h2 Обо мне
          %p
          %i Горбенко Кирилл Николаевич.
          Группа ИС-17-2-о. Исследование возможностей
          языка разметки гипертекстов HTML и каскадных таблиц стилей CSS.
          %strong Вариант № 3.
          %p
          = nonsense_paragraph()
      %section
          %header
          %h2 Следующая глава
          %p
          = nonsense_paragraph()
      = render("./partials/footer.haml")
      %script{ src: "../js/index.js" }
\end{lstlisting}

\subsubsection{Страница \enquote{Обо мне}}
Текст документа \code{about.haml} приведен далее:
\begin{lstlisting}
%html
  %head
    %title About
    = render("./partials/head-shared.haml")
  %body
    = render("./partials/nav.haml")
    = render("./partials/portrait.haml")
    %div.main-content
      %header
        %h1 Автобиография
      %section
        %header
          %h1 Раздел 1
        %p Здесь должна быть какая-то автобиография. Пускай тут будет списочек:
        %ol
          %li 19 лет.
          %li Сын своих родителей.
          %li Не знаю чего написать еще.
          %li Пойду заварю чайку.
          %li Заварил.
          %li Рыба меч дыня друг.
        %p Вот еще фоточки:
        %div.photo-wrapper
          - about_page_photos().each do |photo|
            %img{ src: "#{photo[:src]}", alt: "#{photo[:alt]}"}
    = render("./partials/footer.haml")
    %script{ src: "../js/about.js" }
\end{lstlisting}

\subsubsection{Страница \enquote{Мои интересы}}
Текст документа \code{interests.haml} приведен далее:
\begin{lstlisting}
%html
  %head
    %title My interests
    = render("./partials/head-shared.haml")
    %link{ href: "../src/modal/modal.css", rel: "stylesheet", type: "text/css" }
  %body
    = render("./partials/nav.haml")
    = render("./partials/portrait.haml")
    %div.main-content
      %header
        %h1 Мои интересы
      %section
        %header
          %h2 Contents
        %button#dialog.modal-button Modal
        %ul#interests
          %li.contents#Hobbies Мои хобби
          %li#Hobbies-expand.interests
            %h2 My hobbies
            %p My hobbies are supposed to be here
            %p
              = nonsense_paragraph()
          %li.contents#Books Мои книги
          %li#Books-expand.interests
            %h2 My books
            %p My books are supposed to be here
            %p
              = nonsense_paragraph()
          %li.contents#Music Моя музыка
          %li#Music-expand.interests
            %h2 My music
            %p My music is supposed to be here
            %p
              = nonsense_paragraph()
          %li.contents#Films Мои фильмы
          %li#Films-expand.interests
            %h2 My films
            %p My films are supposed to be here
            %p
              = nonsense_paragraph()
    = render("./partials/footer.haml")
    %div.modal
      %div.modal-window
        %div
          %p.modal-message
        %div.modal-navigation
          %button.modal-button.modal-confirm-button
          %button.modal-button.modal-cancel-button
    %script{ src: "../js/interests.js" }
\end{lstlisting}

\subsubsection{Страница \enquote{Учеба}}
Текст документа \code{learning.haml} приведен далее:
\begin{lstlisting}
%html
  %head
    %title M university disciplines page
    = render("./partials/head-shared.haml")
  %body
    = render("./partials/nav.haml")
    = render("./partials/portrait.haml")
    %div.main-content
    %section
        %header
        %h1 Дисциплины
        %p Институт информационных технологий и управления в технических системах
        %p Кафедра информационных систем
        %table
        %tr
            %td{:colspan => "9"} Дисциплины по семестрам
        %tr
            %th{:scope => "row"} 1 семестр
            %td Высшая математика
            %td Алгоритмизация и программирование
            %td Информатика
            %td История
            %td Физика
            %td Экология
            %td Дискретная математика
            %td Английский язык
        %tr
            %th{:scope => "row"} 2 семестр
            %td Высшая математика
            %td Алгоритмизация и программирование
            %td Физика
            %td Философия
            %td Основы теории права
            %td Экономика
            %td БЖД
            %td Английский язык
        %tr
            %th{:scope => "row"} 3 семестр
            %td Архитектура компьютерных систем
            %td Высшая математика
            %td Дискретная математика
            %td ООП
            %td ТЭЦ
            %td Экономика предприятия
            %td Компьютерная графика
            %td Английский язык
        %tr
            %th{:scope => "row"} 4 семестр
            %td Электроника
            %td Основы теории алгоритмов
            %td Теория вероятностей и математическая статистика
            %td Английский язык
            %td Системный анализ
            %td Теория баз данных
            %td ТСПП
            %td ОС
    = render("./partials/footer.haml")
    %script{ src: "../js/learning.js" }
\end{lstlisting}

\subsubsection{Страница \enquote{Фотоальбом}}
Текст документа \code{photos.haml} приведен далее:
\begin{lstlisting}
%html
  %head
      %title Photo album page
      = render("./partials/head-shared.haml")
      %link{ href: "../src/lightbox/lightbox.css", rel: "stylesheet", type:"text/css" }
  %body
      = render("./partials/nav.haml")
      = render("./partials/portrait.haml")
      %div.main-content
      %header
          %h1 Фотоальбом
      %div.photo-wrapper
      = render("./partials/footer.haml")
      %div#lightbox
      %div.lightbox-content
          %div.image-wrapper
          %img.lightbox-picture{ src: "file:///D:/Learning/Web/Portfolio/images/lion-king.jpg" }
          %div.lightbox-navigation
          %div.lightbox-button.lightbox-arrow-left &lt;
          %div.lightbox-button.lightbox-arrow-right &gt;
      %div.lightbox-button.lightbox-close ×
      %script{ src: "../js/photos.js" }
\end{lstlisting}

\subsubsection{Страница \enquote{Контакт}}
Текст документа \code{contact.haml} приведен далее:
\begin{lstlisting}
%html
  %head
      %title Contact me
      = render("./partials/head-shared.haml")
      %link{ href: '../src/datepicker/datepicker.css', rel: "stylesheet", type: "text/css" }
      %link{ rel: "stylesheet", type: "text/css", href: "../node_modules/tooltipster/dist/css/tooltipster.bundle.min.css" }
      %link{ rel: "stylesheet", type: "text/css", href: "../node_modules/tooltipster/dist/css/plugins/tooltipster/sideTip/themes/tooltipster-sideTip-light.min.css" }
  %body
      = render("./partials/nav.haml")
      = render("./partials/portrait.haml")
      %div.main-content
      %section
          %header
          %h1 Mail Me!
          %p All fields are required! Hover over fields to see the additional requirements.
          %form{ action: "someaction", method: "POST" }
          %ol
              %li
              %label{ for: "full-name-input"} Full Name
              %input#full-name-input.field-long{ name: "full-name", type: "text" }
              %br
              %li
              %label{ for: "email-input" } Email
              %input#email-input.field-long{ name: "email", type: "email"}
              %br
              %li
              %label{ for: "phone-input" } Phone number
              %input#phone-input.field-long{ name: "phone", type: "text" }
              %br
              %li
              %label{ for: "message-input" } Your Message
              %textarea#message-input.field-long{ name: "field4" }
              %li
              %label{ for: "datepicker-input" } Date
              %input#datepicker-input.field-long{ name: "date", type: "text" }
              %br
              %li
              %input{ type: "submit", value: "Submit"}
              %br  
      = render("./partials/footer.haml")
    %script{ src: "../js/contact.js" }
\end{lstlisting}

\subsubsection{Страница \enquote{Тест}}
Текст документа \code{test.haml} приведен далее:
\begin{lstlisting}
%html
  %head
    %title Test page
    = render("./partials/head-shared.haml")
    %link{ href: '../src/datepicker/datepicker.css', rel: "stylesheet", type: "text/css" }
    %link{ rel: "stylesheet", type: "text/css", href: "../node_modules/tooltipster/dist/css/tooltipster.bundle.min.css" }
    %link{ rel: "stylesheet", type: "text/css", href: "../node_modules/tooltipster/dist/css/plugins/tooltipster/sideTip/themes/tooltipster-sideTip-light.min.css" }
  %body
    = render("./partials/nav.haml")
    = render("./partials/portrait.haml")
    %div.main-content
      %section
        %header
          %h1 Тест по инженерной графике
        %p All fields are required! Hover over fields to see the additional requirements.
        %form{ action: "someaction", method: "POST" }
          %ol
            %li
              %p От чего зависит величина стрелок размерной линии?
              %label.inline{ for: "1" } От длины размерной линии
              %input#1{ checked: "true", name: "question1", type: "radio", value: "1"}
              %br
              %label.inline{ for: "2" } От толщины линии контура изображения
              %input#2{ name: "question1", type: "radio", value: "2" }
              %br
            %li
              %label{ for: "select-input" } Какое назначение имеет тонкая сплошная линия?
              %select#select-input{ name: "question2" }
                %option{ value: "1" } Линии разграничения вида и разреза.
                %option{ value: "2" } Линии сечений.
                %option{ value: "3" } Линии штриховки.
            %li
              %label{ for: "question3" } Какие размеры являются рабочими?
              %textarea#question3.field-long{ name: "question3" }
            %li
              %label{ for: "full-name-input" } Full Name
              %input#full-name-input.field-long{ name: "full-name", type: "text" }
              %br
            %li
              %label{ for: "email-input" } Email
              %input#email-input.field-long{ name: "email", type: "email" }
              %br
            %li
              %label{ for: "phone-input" } Phone number
              %input#phone-input.field-long{ name: "phone", type: "text" }
              %br
            %li
              %label{ for: "message-input" } Your Message
              %textarea#message-input.field-long{ name: "field4" }
            %li
              %input{ type: "submit", value: "Submit"}
              %br
    = render("./partials/footer.haml")
    %script{ src: "../js/test.js" }
\end{lstlisting}

\subsubsection{Страница \enquote{История}}
Текст документа \code{history.haml} приведен далее:
\begin{lstlisting}
%html
  %head
    %title History page
    = render("./partials/head-shared.haml")
  %body
    = render("./partials/nav.haml")
    = render("./partials/portrait.haml")
    %div.main-content
      %header
        %h1 История
      %section
        %header
          %h2 История сессии
        %table.session-history
          %tr
            %th{ colspan: "2" } История сессии
          %tr
            %td Домашняя страница
            %td.home-page
          %tr
            %td Обо мне
            %td.about-page
          %tr
            %td Мои интересы
            %td.interests-page
          %tr
            %td Учеба
            %td.learning-page
          %tr
            %td Фото
            %td.photos-page
          %tr
            %td Контакт
            %td.contact-page
          %tr
            %td Тест
            %td.test-page
          %tr
            %td История
            %td.history-page
      %section
        %header
          %h2 История за все время
        %table.global-history
          %tr
            %th{ colspan: "2" } История за все время
          %tr
            %td Домашняя страница
            %td.home-page
          %tr
            %td Обо мне
            %td.about-page
          %tr
            %td Мои интересы
            %td.interests-page
          %tr
            %td Учеба
            %td.learning-page
          %tr
            %td Фото
            %td.photos-page
          %tr
            %td Контакт
            %td.contact-page
          %tr
            %td Тест
            %td.test-page
          %tr
            %td История
            %td.history-page
    = render("./partials/footer.haml")
    %script{ src: "../js/history.js" }
\end{lstlisting}

\subsection{Автоматизация}
Так как с помощь команды можно распарсить только один \code{haml} файл, был создан \code{Powershell} скрипт,
который парсит весь сайт. Текст скрипта приведен далее:

\begin{lstlisting}
param(
    [string] $Dir,
    [string] $Globals
)

$pages = @(
    'index',
    'about',
    'interests',
    'learning',
    'photos',
    'contact',
    'test',
    'history'
)

try {
    $pages | ForEach-Object {
        Write-Host "Processing $_.haml file."

        $hamlFilePath = Join-Path $Dir "$_.haml"
        $htmlFilePath = Join-Path $Dir "$_.html"
        $command = "haml -r $Globals $hamlFilePath $htmlFilePath"
        
        Write-Host "Executing command `n$command"

        Invoke-Expression $command
        if (-not $?) { throw "An error occured during executing command $command." }

        Write-Host "Command successfully executed. Output file is $htmlFilePath."
    }
    exit 0
} catch {
    Write-Error $_.Exception.Message
    exit 1
}
\end{lstlisting}

\section*{Выводы}
В ходе лабораторной работы для создания \code{html} документа страниц сайта использовался
препроцессор \code{haml}. С его помощью можно ускорить разработку веб-страниц за счет
уменьшния объема разрабатываемого кода и автоматизации таких вещей как закрытие тегов.

Кроме того, \code{haml} содержит функции, недоступные при создании обычного \code{html}:
циклы и ветвления, что также упрощает разработку.

В среде отличной от \code{Ruby on Rails} использовать \code{haml} я бы не стал из-за сильной
привязки препроцессора к инфраструктуре языка \code{Ruby}. В таком случае получается, что
появляется необходимость расширять \code{haml} компонентами, написанными на неосновном
языке разработки.

\end{document}