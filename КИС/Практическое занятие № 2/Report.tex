\documentclass[a4paper,14pt]{extarticle}
\usepackage{../../tex-shared/report-layout}

\begin{document}
\begin{titlepage}
    
    \thispagestyle{empty}
    
    \begin{center}
        
        Министерство науки и Высшего образования Российской Федерации \\
        Севастопольский государственный университет \\
        Кафедра ИС
        
        \vfill

        Отчет \\
        по лабораторной работе №\mylabnumber \\
        \enquote{\mylabtitle} \\
        по дисциплине \\
        \enquote{\MakeTextUppercase{\mysubject}}

    \end{center}

    \vspace{1cm}

    \noindent\hspace{7.5cm} Выполнил студент группы ИС/б-17-2-о \\
    \null\hspace{7.5cm} Горбенко К. Н. \\
    \null\hspace{7.5cm} Проверил \\
    \null\hspace{7.5cm} \mylecturer

    \vfill

    \begin{center}
        Севастополь \\
        \the\year{}
    \end{center}

\end{titlepage}

\section{Программа курса}
В рамках курса вы изучите технологии и методы организации распределенных
параллельных вычислений, методов распараллеливания последовательных алгоритмов
обработки данных, особенности их программной реализации. Технологическим
средством, рассматриваемым в данном курсе, является библиотека MPI, позволяющая
создавать параллельные распределенные приложения, и являющаяся стандартом в этой
области.

Под MPI подразумевается модель реализации задач, взаимодействующих посредством
параллельной передачи сообщений, в которой через сов-местные операции над каждым
из процессов данные перемещаются из ад-ресного пространства одного процесса в
адресное пространство другого. MPI не является языком программирования, все
операции MPI выражаются в виде функций, подпрограмм, или методов с
соответствующими привяз-ками к языку (C/C++, Fortran-77, 95).

В рамках курса будут рассмотрены следующие темы:
\begin{enumerate}
    \item Классификация параллельных вычислительных систем. Организация конвейерных систем, реализация параллельных вычислений в конвейерных системах.
    \item Организация матричных вычислительных систем, реализация параллельных вычислений в матричных системах.
    \item Организация и функционирование многопроцессорных систем с общей и разделенной памятью, реализация параллельных вычислений в многопроцессорных системах с общей и разделенной памятью.
    \item Основные понятия распределенного программирования. Модели взаимодействия между параллельными процессами. Организация взаимодействия распределенных процессов по схеме \enquote{взаимодействующие равные}. Алгоритмы взаимодействия процессов.
    \item Взаимодействие распределенных процессов по схеме \enquote{клиент-сервер}. Схемы организации взаимодействия. Алгоритмы функционирования сервера.
    \item Параллельные алгоритмы сортировки данных.
    \item Параллельные алгоритмы определения кратчайших путей на графах.
\end{enumerate}

\end{document}