\documentclass[a4paper,14pt]{extarticle}
\usepackage{../../tex-shared/no-title-layout}

\begin{document}

\section{ВВЕДЕНИЕ В WPF}
\subsection{История}
Базовые технологии большинства интерфейсов в Windows – интерфейс графического
устройства (Graphics Device Interface, GDI) и подсистема USER – появились
в Windows 1.0 еще в 1985 году. В начале 1990-х годов компания Silicon
Graphics разработала ставшую популярной графическую библиотеку OpenGL
для двумерной и трехмерной графики как в Windows, так и в других системах.
Она была с восторгом принята компаниями, работающими в сфере создания
систем автоматизированного проектирования, программ визуализации научных данных и игр.
Технология Microsoft DirectX, представленная в 1995 году, обеспечила высокоскоростную
альтернативу для 2D-графики, ввода, сетевого взаимодействия, работы со звуком, а со
временем и 3D-графики (которая стала возможной с версией DirectX 2, вышедшей в 1996 году).

Впоследствии и в GDI, и в DirectX было внесено много существенных улучшений. Например,
технология GDI+, представленная в Windows XP, добавила поддержку прозрачности и
градиентные кисти. Однако ввиду большой сложности и в отсутствие аппаратного ускорения
она работает медленнее, чем GDI. 

После появления каркаса .NET и управляемого кода (в 2002 году) разработчики получили
очень продуктивную модель для создания Windows и веб-приложений. Включенная в нее
технология Windows Forms (основанная на GDI+) стала основным способом создания
пользовательских интерфейсов в Windows для разработчиков на C\#, Visual Basic
и (в меньшей степени) С++. Она пользовалась успехом и оказалась весьма продуктивной,
но имела фундаментальные ограничения, уходящие корнями в GDI+ и подсистему USER.

Графические подсистемы компьютеров продолжали совершенствоваться и дешеветь, ожидания
потребителей росли, но до появления WPF проблеме сложности построения выразительных
пользовательских интерфейсов не уделяли должного внимания. Некоторые разработчики
самостоятельно брались за ее решение, стремясь сделать свои приложения и элементы управления
более привлекательными. Простым примером тут является использование растровых изображений
вместо стандартных кнопок. Однако мало того что подобные нестандартные решения было трудно
реализовывать, они еще зачастую оказывались ненадежными. Основанные на них приложения не
всегда доступны людям с ограниченными возможностями, плохо адаптируются к высокому разрешению
и имеют другие визуальные огрехи.

Корпорация Microsoft понимала, что необходимо нечто совершенно новое, свободное от
ограничений GDI+ и подсистемы USER, но не менее продуктивное и удобное в использовании,
чем Windows Forms. И с учетом роста популярности кроссплатформенных приложений, основанных
на HTML и JavaScript, Windows отчаянно нуждалась в столь же простой технологии, которая
при этом еще и позволяла бы задействовать все возможности локального компьютера. И Windows
Presentation Foundation (WPF) дала в руки разработчиков ПО и графических дизайнеров тот
инструмент, который был необходим для создания современных решений и не требовал освоения
сразу нескольких сложных технологий.

\subsubsection*{Основные возможности, которые предоставляет WPF}

\begin{itemize}
    \item \textbf{Широкая интеграция.} До WPF разработчикам в Windows, которые хотели использовать
    одновременно 3D-графику, видео, речь и форматированные документы в дополнение к обычной
    двумерной графике и элементам управления, приходилось изучать несколько независимых
    технологий, плохо совместимых между собой и не имеющих встроенных средств сопряжения.
    А в WPF все это входит в состав внутренне согласованной модели программирования,
    поддерживающей композицию и визуализацию разнородных элементов.

    \item \textbf{Независимость от разрешающей способности.} Только представьте себе
    мир, в котором переход к более высокому разрешению экрана или принтера не означает,
    что все уменьшается. Вместо этого графические элементы и текст только становятся более
    четкими! Представьте себе пользовательский интерфейс, который прекрасно выглядит и на
    маленьком нетбуке, и на полутораметровом экране телевизора! WPF все это обеспечивает
    и дает возможность уменьшать или увеличивать элементы на экране независимо от его
    разрешения. Это стало возможным благодаря тому, что WPF основана на использовании
    векторной графики.

    \item \textbf{Аппаратное ускорение.} Поскольку WPF основана на технологии Direct3D,
    то все содержимое в WPF-приложении, будь то двумерная или трехмерная графика,
    изображения или текст, преобразуется в трехмерные треугольники, текстуры и другие
    объекты Direct3D, а потом отрисовывается аппаратной графической подсистемой компьютера.
    Таким образом, WPF-приложения задействуют все возможности аппаратного ускорения графики,
    что позволяет добиться более качественного изображения и одновременно повысить
    производительность (поскольку часть работы перекладывается с центральных процессоров
    на графические). При этом от применения новых графических ускорителей и их драйверов
    выигрывают все WPFприложения (а не только высококлассные игры). Но WPF не требует
    обязательного наличия высокопроизводительной графической аппаратуры.

    \item \textbf{Декларативное программирование.} Декларативное программирование не
    является уникальной особенностью WPF, поскольку в программах на платформе
    Win16/Win32 сценарии описания ресурсов для определения компоновки диалоговых окон и меню
    применяются вот уже 25 лет. И в .NET-приложениях часто используются декларативные
    атрибуты наряду с конфигурационными и ресурсными XML-файлами. Однако в WPF применение
    декларативного программирования вышло на новый уровень благодаря языку XAML (eXtensible
    Application Markup Language – расширяемый язык разметки приложений). Сочетание WPF и XAML
    аналогично использованию HTML для описания пользовательского интерфейса, но с гораздо более
    широкими выразительными возможностями. И эта выразительность выходит далеко за рамки
    описания интерфейса. В WPF язык XAML применяется в качестве формата документов, для
    представления 3D-моделей и многого другого.

    \item \textbf{Богатые возможности композиции и настройки.} В WPF элементы управления могут
    сочетаться немыслимыми ранее способами. Можно создать комбинированный список, содержащий
    анимированные кнопки, или меню, состоящее из видеоклипов! Конечно, сама мысль о таком интерфейсе
    может привести в ужас, но важно то, что для оформления элемента способом, о котором его автор
    и не помышлял, не понадобится писать никакой (!) код (и в этом коренное отличие от предшествующих
    технологий, где отрисовка элементов жестко задавалась создателем кода).
\end{itemize}

\end{document}