\documentclass[a4paper,14pt]{extarticle}
\usepackage{../../tex-shared/report-layout}

\renewcommand{\mylabnumber}{5}
\renewcommand{\mylabtitle}{Исследование способов модульного тестирования
программного обеспечения в среде NUnit}
\renewcommand{\mysubject}{Тестирование ПО}
\renewcommand{\mylecturer}{Тлуховская Н.П.}

\begin{document}
\begin{titlepage}
    
    \thispagestyle{empty}
    
    \begin{center}
        
        Министерство науки и Высшего образования Российской Федерации \\
        Севастопольский государственный университет \\
        Кафедра ИС
        
        \vfill

        Отчет \\
        по лабораторной работе №\mylabnumber \\
        \enquote{\mylabtitle} \\
        по дисциплине \\
        \enquote{\MakeTextUppercase{\mysubject}}

    \end{center}

    \vspace{1cm}

    \noindent\hspace{7.5cm} Выполнил студент группы ИС/б-17-2-о \\
    \null\hspace{7.5cm} Горбенко К. Н. \\
    \null\hspace{7.5cm} Проверил \\
    \null\hspace{7.5cm} \mylecturer

    \vfill

    \begin{center}
        Севастополь \\
        \the\year{}
    \end{center}

\end{titlepage}
\section{Цель работы}
Исследовать эффективность использования методологии TDD при разработке
программного обеспечения. Получить практические навыки использования
фреймворка NUnit для модульного тестирования программного обеспечения.

\section{Порядок выполнения работы}
\begin{enumerate}
    \item Реализовать на языке С\# один из классов, спроектированных в лабораторной
    работе № 1. Методы класса при этом не реализовывать.
    \item Разработать для созданного класса набор модульных тестов,
    включающий тесты для каждого метода.
    \item Запустить набор тестов, проанализировать и сохранить результаты.
    \item Поочередно реализовать методы класса, выполняя тестирование при
    каждом изменении программного кода.
    \item После того, как весь набор тестов будет выполняться успешно,
    реализацию классов можно считать завершенной.
\end{enumerate}

\section{Ход работы}
Для тестирования выберем класс \code{MatrixOperations}:
\begin{lstlisting}
public static class MatrixOperations
    {
        public static IEnumerable<int> GetSumsOfColumns(int[,] matrix)
        {
            throw new NotImplementedException();
        }
    }
\end{lstlisting}

Класс содержит один метод, в обязанности которого входит вычисление сумм 
положительных элементов в столбцах матрицы. При этом, если передать в метод
неквадрутную матрицу, то происходит ошибка.

Создадим тестовые случаи:
\begin{enumerate}
    \item Матрица - null. \\
          Ожидаемый результат - ошибка.
    \item Передается неквадратная матрица. \\
          Ожидаемый результат - ошибка.
    \item Матрица состоит из одного положительного элемента.\\
          Ожидаемый результат - массив с одним элементом.
    \item Матрица состоит из одного отрицательного элемента. \\
          Ожидаемый результат - массив с одним элементом (0).
    \item Квадратная матрица без отрицательных элементов. \\
          Ожидаемый результат - массив с суммами всех элементов по столбцам.
    \item Квадратная матрица с отрицательными элементами. \\
          Ожидаемый результат - массив с суммами всех элементов по столбцам без учета отрицательных.
\end{enumerate}

Опишем тесты:
\begin{lstlisting}
public class MatrixOperationsTests
{
    [Test]
    public void PassingNullIssuesException()
    {
        Assert.Throws<ArgumentNullException>(() => GetSumsOfPositiveElementsInColumns(null));
    }

    [Test]
    public void PassingNonSquareMatrixIssuesException()
    {
        var matrix = new[,] {{1, 3, 5}, {2, 3, 4}};
        
        Assert.Throws<ArgumentException>(() => GetSumsOfPositiveElementsInColumns(matrix));
    }

    [Test]
    public void SingleNegativeItem()
    {
        var matrix = new[,] {{-15}};
        var expected = new[]{0};
        
        var actual = GetSumsOfPositiveElementsInColumns(matrix);

        Assert.That(expected, Is.EqualTo(actual));
    }

    [Test]
    public void SinglePositiveItem()
    {
        var matrix = new[,] {{3}};
        var expected = new[] {3};

        var actual = GetSumsOfPositiveElementsInColumns(matrix);
        
        Assert.That(expected, Is.EqualTo(actual));
    }

    [Test]
    public void SquareMatrixWithoutNegativeElements()
    {
        var matrix = new[,]
        {
            {15, 10, 36},
            {13, 0, 14},
            {62, 15, 1235}
        };
        var expected = new[] {90, 25, 1285};

        var actual = GetSumsOfPositiveElementsInColumns(matrix);

        Assert.That(expected, Is.EqualTo(actual));
    }

    [Test]
    public void SquareMatrixWithNegativeElements()
    {
        var matrix = new[,]
        {
            {-30, 10, 36},
            {13, 0, 14},
            {62, -5, 1235}
        };
        var expected = new[] {75, 10, 1285};

        var actual = GetSumsOfPositiveElementsInColumns(matrix);

        Assert.That(expected, Is.EqualTo(actual));
    }
}
\end{lstlisting}

Реализуем класс:
\begin{lstlisting}
public static class MatrixOperations
{
    public static IEnumerable<int> GetSumsOfPositiveElementsInColumns(int[,] matrix)
    {
        if (matrix == null)
        {
            throw new ArgumentNullException(nameof(matrix));
        }

        if (matrix.GetLength(0) != matrix.GetLength(1))
        {
            throw new ArgumentException("Only square martixes are allowed.");
        }

        return matrix
                .GetColumns()
                .Select(column => column.Where(x => x > 0).Sum());
    }
}
\end{lstlisting}

\section*{Выводы}
В ходе лабораторной работы было выполнено тестирование класса с помощью фреймворка NUnit
с использованием подхода TDD. TDD предполагает обдумывание реализации модулей и написание
тестов к ним перед написанием самих модулей. Это позволяет описать функционал модуля
таким образом, чтобы при дальнейшей его модификации не допустить таких изменений,
которые повлияют на основной функционал.

Кроме того, модуль организовывается таким образом, чтобы удовлетворять лишь требованиям,
возлагаемым на низх тестами. Таким образом, меньше времени тратится непосредственно на
разработку, так как считается, что программист обдумал детали реализации еще при написании
тестов.
\end{document}