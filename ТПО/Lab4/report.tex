\documentclass[a4paper,14pt]{extarticle}
\usepackage{../../tex-shared/report-layout}

\renewcommand{\mylabnumber}{4}
\renewcommand{\mylabtitle}{Исследование способов интеграционного
тестирования программного обеспечения}
\renewcommand{\mysubject}{Тестирование ПО}
\renewcommand{\mylecturer}{Тлуховская Н.П.}

\begin{document}
\begin{titlepage}
    
    \thispagestyle{empty}
    
    \begin{center}
        
        Министерство науки и Высшего образования Российской Федерации \\
        Севастопольский государственный университет \\
        Кафедра ИС
        
        \vfill

        Отчет \\
        по лабораторной работе №\mylabnumber \\
        \enquote{\mylabtitle} \\
        по дисциплине \\
        \enquote{\MakeTextUppercase{\mysubject}}

    \end{center}

    \vspace{1cm}

    \noindent\hspace{7.5cm} Выполнил студент группы ИС/б-17-2-о \\
    \null\hspace{7.5cm} Горбенко К. Н. \\
    \null\hspace{7.5cm} Проверил \\
    \null\hspace{7.5cm} \mylecturer

    \vfill

    \begin{center}
        Севастополь \\
        \the\year{}
    \end{center}

\end{titlepage}
\section{Цель работы}
Исследовать основные принципы интеграционного тестирования
программного обеспечения. Приобрести практические навыки организации
интеграционных тестов для объектно-ориентированных программ.

\section{Порядок выполнения работы}
\begin{enumerate}
    \item Выбрать в качестве тестируемого взаимодействие двух или более
          классов, спроектированных в лабораторных работах № 1 – 4.
    \item Составить спецификацию тестового случая.
    \item Реализовать тестируемые классы и необходимое тестовое
          окружение на языке С\#.
    \item Выполнить тестирование с выводом результатов на экран и
          сохранением в log-файл.
    \item Проанализировать результаты тестирования, сделать выводы.
\end{enumerate}

\section{Ход работы}
\sloppy
В качестве тестируемого взаимодействия выберем взаимодействие классов
\code{GetColumnsUnitTestEngine}, \code{FileLogger} и
\code{GetColumnsUnitTestDto}, где \code{GetColumnsUnitTestEngine} -
тестовый двигатель для выполнения модульных тестов над методом \code{GetColumns}
(Л.Р.№ 2).
\begin{lstlisting}
public class GetColumnsUnitTestEngine
{
    private ILogger Logger { get; set; }
    private IEnumerable<GetColumnsUnitTestDto> TestSuite { get; set; }

    public GetColumnsUnitTestEngine(ILogger logger, IEnumerable<GetColumnsUnitTestDto> testSuite)
    {
        Logger = logger ?? throw new ArgumentNullException(nameof(logger));
        TestSuite = testSuite ?? throw new ArgumentNullException(nameof(logger));
    }

    public void Run()
    {
        foreach (var test in TestSuite)
        {
            try
            {
                var actual = test.Input.GetColumns();
                var result = actual.SequenceEqual(test.Expected, new Int32ArrayEqualityComparer());

                Logger.Log($"Test {test.Name}," + 
                            $"result:{result}" + 
                            $"{(result ? "" : $",expected: {PrintArrayCollection(test.Expected)} actual:{PrintArrayCollection(actual)}")}");
            }
            catch (Exception e)
            {
                Logger.Log($"Test {test.Name}," +
                            $"result:{false}," +
                            $"Exception: {e.Message}");
            }
        }
    }

    private string PrintArrayCollection(IEnumerable<int[]> collection)
    {
        var stringBuilder = new StringBuilder("{ ");

        foreach (var array in collection)
        {
            stringBuilder.Append("[ ");
            foreach (var item in array)
            {
                stringBuilder.Append($"{item} ");
            }
            stringBuilder.Append("] ");
        }
        stringBuilder.Append("}");
        return stringBuilder.ToString();
    }
}
\end{lstlisting}

Класс \code{GetColumnsUnitTestEngine} зависит от экземпляров \code{FileLogger} 
и \code{GetColumnsUnitTestDto}. Это взаимодействие 3-го типа (классы \code{FileLogger} и
\code{GetColumnsUnitTestDto} - часть его реализации).

\subsection{Тестовые случаи}
Опишем два тестовых случая:
\begin{enumerate}
    \item Тест экземпляра \code{FileLogger}
    \begin{enumerate} 
        \item Взаимодействующие классы: \code{GetColumnsUnitTestEngine} и \code{FileLogger}.
        \item Имя теста: \code{GetColumnsUnitTestEngine} вызывает метод \code{Log} класса
              \code{FileLogger}.
        \item Описание теста: при прогоне тестовой последовательности, класс
              \code{GetColumnsUnitTestEngine} для записи реультатов должен вызывать
              метод \code{Log} класса \code{FileLogger}.
    \end{enumerate}

    \item Тест экземпляров тестов
    \begin{enumerate}
        \item Взаимодействующие классы: \code{GetColumnsUnitTestEngine} и \code{GetColumnsUnitTestDto}.
        \item Имя теста: \code{GetColumnsUnitTestEngine} записывает результат выполнения каждого теста
              последовательности в файл.
        \item Описание теста: после прогона тестовой последовательности, для каждого теста в тестовой
              последовательности должна остаться запись в файле. 
    \end{enumerate}
\end{enumerate}

\section{Программная реализация}
Для тестирования случая № 1 создадим замену классу \code{FileLogger}:
\begin{lstlisting}
public class FakeLogger : ILogger
{
    public bool LogCalled { get; private set; } = false;

    public void Log(string message)
    {
        LogCalled = true;
    }

    public void Dispose() { }
}
\end{lstlisting}

Свойство \code{LogCalled} позволяет определить, был ли метод \code{Log} вызван.
Использование замены:
\begin{lstlisting}
private bool TestThatLogMethodGetsCalled()
{
    try
    {
        this.logger.Log($"Test {nameof(TestThatLogMethodGetsCalled)}.");
        var logger = new FakeLogger();
        var unitTestEngine = new GetColumnsUnitTestEngine(logger, UnitTestConstants.TestSuite);
        this.logger.Log("Passing logger.");
        this.logger.Log("Running unit tests.");
        unitTestEngine.Run();
        bool result = logger.LogCalled;
        if (result == true)
        {
            this.logger.Log("Test passed.");
        }
        else
        {
            this.logger.Log("Test failed.");
        }
        return result;
    }
    catch
    {
        return false;
    }
}
\end{lstlisting}

Для тестирования случая № 2 выполним последовательно проверку каждой строки файла.
Она должна содержать имя прогоняемого теста. Реализация:
\begin{lstlisting}
private bool TestThatAllTestSuiteTestsAreRunning()
{
    try
    {
        this.logger.Log($"Test {nameof(TestThatAllTestSuiteTestsAreRunning)}.");
        var logger = new FileLogger(new StreamWriter(UnitTestConstants.Path));
        var unitTestEngine = new GetColumnsUnitTestEngine(logger, UnitTestConstants.TestSuite);
        this.logger.Log("Passing logger.");
        this.logger.Log("Running unit tests.");
        unitTestEngine.Run();
        logger.Dispose();
        this.logger.Log("Checking file content.");
        bool result = true;
        using (var streamReader = new StreamReader(UnitTestConstants.Path))
        {
            foreach (var test in UnitTestConstants.TestSuite)
            {
                var fileLine = streamReader.ReadLine();
                if (!fileLine.StartsWith($"Test {test.Name}"))
                {
                    result = false;
                }
            }
        }
        if (result == true)
        {
            this.logger.Log("Test passed.");
        }
        else
        {
            this.logger.Log("Test failed.");
        }
        return result;
    }
    catch
    {
        return false;
    }
}
\end{lstlisting}

Тестовый драйвер полученных интеграционных тестов:
\begin{lstlisting}
public class GetColumnsUnitTestEngineIntegrationTestEngine
{
    private readonly ILogger logger;
    
    public GetColumnsUnitTestEngineIntegrationTestEngine(ILogger logger)
    {
        this.logger = logger ?? throw new ArgumentNullException(nameof(logger));
    }
    
    public void Run()
    {
        TestThatLogMethodGetsCalled();
        TestThatAllTestSuiteTestsAreRunning();
    }

    private bool TestThatLogMethodGetsCalled()
    {
        ...
    }

    private bool TestThatAllTestSuiteTestsAreRunning()
    {
        ...
    }
}
\end{lstlisting}

\section*{Выводы}
В ходе лабораторной работы был протестирован тестовый драйвер, созданный
для реализации модульных тестов к лабораторной работе № 2. Тестирование
проводилось с точки зрения интеграции в него экземпляров объектов логгера
и тестовой последовательности. Было выявлено, что тестовый драйвер правильно
взаимодействует с используемыми объектами (в соответствии с их спецификацией).

Модульное и интеграционное тестирование представляет из себя трудоемкий процесс
из-за необходимости создавать тестовое окружение для каждого тестируемого метода.
Чтобы упростить этот процесс необходимо использовать тестовые и mock-фреймворки. 
\end{document}