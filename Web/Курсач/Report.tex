\documentclass[a4paper,14pt]{extarticle}
\usepackage{../../tex-shared/no-title-layout}

\begin{document}

\section{ТЕХНИЧЕСКОЕ ЗАДАНИЕ}
\subsection{Цели создания и целевая аудитория}
Целью создания сайта является помощь в самостоятельном изучении английского языка
лицам, желающим укрепить свой уровень знаний.

Целевой аудиторией сайта являются люди от 10 лет, имеющие базовые навыки работы с
компьютером и знания основ английского языка.

\subsection{Структура WEB-сайта}
\subsubsection{Разметка страницы}
Любая страница сайта должна включать в себя фиксированный хедер с панелью навигации, логотипом сайта и панелью аутентификации.
Под хедером должен располагаться контент конкретной страницы. В нижней части сайта должен находиться футер с информацией о сайте,
ссылками на социальные сети, ссылкой на Github, логотипом сайта, контактной информацией для отзывов и предложений.
\subsubsection{Лендинговая страница}
Лендинговая страница представляет собой страницу с панелью навигации, кнопками \enquote{Войти}
и \enquote{Зарегистрироваться}, а также кратким описанием возможностей проекта.

В навигационной панели на этой странице ссылки в меню отображаться не должны.

\subsubsection{Страница регистрации}
На странице регистрации должна находиться форма со следующими полями:
\begin{enumerate}
    \item логин;
    \item пароль;
    \item подтверждение пароля;
    \item электронная почта;
    \item проверка на робота;
    \item согласие на обработку персональных данных.
\end{enumerate}

\subsubsection{Страница аутентификации}
На странице регистрации должна находиться форма со следующими полями:
\begin{enumerate}
    \item логин;
    \item пароль.
\end{enumerate}

\subsubsection{Домашняя страница}
На домашней странице должна находиться информация о пользователе: последние пройденные уроки,
средний балл, информация по личному словарю (количество добавленных слов, выученных слов, невыученных слов),
общее количество потраченных часов на изучение, таблицы с доступными тестами и лекциями по темам,
пройденные тесты и лекции. 

\subsubsection{Страница лекций}
Страница содержит список всех лекций.

\subsubsection{Страница тестов}
Страница содержит список всех тестов.

\subsubsection{Страница лекции}
На странице урока обучающийся может прочитать выбранную лекцию, перейти к соответствующему тесту и личному словарю.

\subsubsection{Страница теста}
На странице теста учащемуся предлагается пройти тестирование по выбранной теме.

\subsubsection{Страница словаря}
На странице содержится список изученных или отмеченных слов, форма добавления нового слова в словарь,
кнопка удаления слова, кнопки вызова теста по изученным словам, по неизученным словам.

\end{document}