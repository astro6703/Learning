\documentclass[a4paper,14pt]{extarticle}
\usepackage{../../tex-shared/report-layout}

\renewcommand{\mylabnumber}{3}
\renewcommand{\mylabtitle}{Исследование рисков для выбранного проекта}
\renewcommand{\mysubject}{Управление IT-проектами}
\renewcommand{\mylecturer}{Смирнова Н.Б.}

\begin{document}
\begin{titlepage}
    
    \thispagestyle{empty}
    
    \begin{center}
        
        Министерство науки и Высшего образования Российской Федерации \\
        Севастопольский государственный университет \\
        Кафедра ИС
        
        \vfill

        Отчет \\
        по лабораторной работе №\mylabnumber \\
        \enquote{\mylabtitle} \\
        по дисциплине \\
        \enquote{\MakeTextUppercase{\mysubject}}

    \end{center}

    \vspace{1cm}

    \noindent\hspace{7.5cm} Выполнил студент группы ИС/б-17-2-о \\
    \null\hspace{7.5cm} Горбенко К. Н. \\
    \null\hspace{7.5cm} Проверил \\
    \null\hspace{7.5cm} \mylecturer

    \vfill

    \begin{center}
        Севастополь \\
        \the\year{}
    \end{center}

\end{titlepage}

\section{Цель работы}
Исследовать риски для выбранного проекта.

\section{Задание на работу}
Для выбранного ранее проекта определить риски, классифицировать их. Продумать и
описать стратегии предотвращения рисков и минимизационные стратегии для 10
\enquote{верхних} рисков.

\section{Ход работы}
\subsection{Риск потерять сотрудников}
Триггеры:
\begin{itemize}
    \item отсутствие участников команды в связи с больничным или отпуском;
    \item уход членов команды.
\end{itemize}

Планы по управлению:
\begin{enumerate}
    \item \textbf{Избегание:}
    \begin{itemize}
        \item обеспечение полных знаний о проекте всеми (большинством) членами команды;
        \item составление документации.
    \end{itemize}

    \item \textbf{Смягчение:}
    \begin{itemize}
        \item замена отсутствующих сотрудников;
        \item организация обучения команды.
    \end{itemize}

    \item \textbf{Передача:}
    \begin{itemize}
        \item существует возможность передать часть работ на исполнение
              сторонним разработчикам (фрилансерам).
    \end{itemize}

    \item \textbf{Аварийная ситуация:} В случае отсутствия значительной части
          команды нет смысла принимать новых сотрудников: их обучение займет
          остальную часть команды. Следует воспользоваться стратегией передачи и
          передать часть работ сторонним разработчикам;
\end{enumerate}

\subsection{отсутствие доступных провайдеров облачных хостинговых услуг для приложений и БД}
Триггеры:
\begin{itemize}
    \item использование технологий, ориентированных на компании энтерпрайз \\уровня;
    \item использование непопулярных технологий.
\end{itemize}

Планы по управлению:
\begin{enumerate}
    \item \textbf{Избегание:}
    \begin{itemize}
        \item использование проверенных массовых технологий при разработке;
    \end{itemize}

    \item \textbf{Смягчение:}
    \begin{itemize}
        \item хостинг приложения и БД на собственном сервере;
    \end{itemize}

    \item \textbf{Передача:} --

    \item \textbf{Аварийная ситуация:} В случае полного отсутствия решений,
          доступных для компании, следует воспользоваться стратегией смягчения и
          завести собственный сервер (сервера) для хостинга приложения и БД.
\end{enumerate}

\subsection{возможность выйти за пределы использования облачного процессорного
времени и объема БД}
Триггеры:
\begin{itemize}
    \item активное использование приложения клиентом после введения в эксплуатацию;
    \item неудачная архитектура приложения или БД.
\end{itemize}

Планы по управлению:
\begin{enumerate}
    \item \textbf{Избегание:}
    \begin{itemize}
        \item тщательное проектирование архитектур приложения или БД;
        \item использование высококлассных сотрудников для проектирования.
    \end{itemize}

    \item \textbf{Смягчение:}
    \begin{itemize}
        \item поиск слабых мест в архитектуре приложения (поиск участков,
              работающих медленно) и их исправление;
        \item поиск слабых мест в архитектуре БД (избыточности данных) и анализ
              возможности устранения избыточности;
    \end{itemize}

    \item \textbf{Передача:} --

    \item \textbf{Аварийная ситуация:} При аварийной ситуации следует
          параллельно с поиском слабых мест в архитектуре приложения с
          привлечением резервных средств перейти на более дорогие планы
          по процессорному времени и объему БД.
\end{enumerate}

\subsection{ошибки при планировании работ по проекту}
Триггеры:
\begin{itemize}
    \item недостаточность проработки календарного плана проекта;
    \item внесение изменений в ТЗ.
\end{itemize}

Планы по управлению:
\begin{enumerate}
    \item \textbf{Избегание:}
    \begin{itemize}
        \item составление гибкого календарного плана.
    \end{itemize}

    \item \textbf{Смягчение:}
    \begin{itemize}
        \item введение резервных дней на каждую работу.
    \end{itemize}

    \item \textbf{Передача:} --

    \item \textbf{Аварийная ситуация:} При невозможности выполнения проекта в
          срок необходимо уведомить об этом клиента; снова спланировать работы с
          учетом новой информации (причин кризиса) и более детальным
          декомпозированием работ.
\end{enumerate}

\subsection{непринятие приложения пользователями}
Триггеры:
\begin{itemize}
    \item разработка неудобного пользовательского интерфейса;
    \item не работает принцип, лежащий в основе приложения.
\end{itemize}

Планы по управлению:
\begin{enumerate}
    \item \textbf{Избегание:}
    \begin{itemize}
        \item изучение потребностей и спроса пользователей, изучение статистики на аналогичных проектах;
        \item привлечение к обсуждению спецификаций проекта экспертов-лингвистов.
    \end{itemize}

    \item \textbf{Смягчение:}
    \begin{itemize}
        \item передача проекта на тестирование реальным пользователям и анализ результата.
    \end{itemize}

    \item \textbf{Передача:}
    \begin{itemize}
        \item Передача разработки дизайна проекта стороннему разработчику.
    \end{itemize}

    \item \textbf{Аварийная ситуация:} --
\end{enumerate}

\subsection{риск несоблюдения технологии}
Триггеры:
\begin{itemize}
    \item использование для реализации проекта новых для команды технологий.
\end{itemize}

Планы по управлению:
\begin{enumerate}
    \item \textbf{Избегание:}
    \begin{itemize}
        \item использование технологий, с которыми команда хорошо знакома.
    \end{itemize}

    \item \textbf{Смягчение:}
    \begin{itemize}
        \item в график проекта необходимо закладывать время на изучение новой
              технологии сотрудниками.
    \end{itemize}

    \item \textbf{Передача:} --

    \item \textbf{Аварийная ситуация:} В случае, когда реализованная на новой
          технологии система не удовлетворяет требований, необходимо новые части
          реализовывать на проверенной технологии; выделить время для
          переписывания уже созданных частей.
\end{enumerate}

\subsection{коммерческие риски}
Триггеры:
\begin{itemize}
    \item увеличение сроков разработки проекта;
    \item изменение финансового состояния заказчика.
\end{itemize}

Планы по управлению:
\begin{enumerate}
    \item \textbf{Избегание:}
    \begin{itemize}
        \item заключение договора, которые предусматривает последовательную
              оплату за каждый этап разработки (итерацию/модуль).
    \end{itemize}

    \item \textbf{Смягчение:}
    \begin{itemize}
        \item Заключение договора по оказанию услуг с точным расчётом их сроков
              и стоимости.
    \end{itemize}

    \item \textbf{Передача:} --

    \item \textbf{Аварийная ситуация:} При неспособности заказчика оплачивать
          работы необходимо выделить такие части функциональности, без которых
          работа проекта невозможна и реализовывать их.
\end{enumerate}

\section*{Выводы}
В ходе выполнения лабораторной работы были изучены риски для проекта сервиса
изучения иностранной лексики. Риски были классифицированы и составлены планы по
управлению рисками.

\end{document}