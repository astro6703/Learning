\documentclass[a4paper,14pt]{extarticle}
\usepackage{../../tex-shared/preamble}

\renewcommand{\mylabnumber}{6}
\renewcommand{\mylabtitle}{Исследование возможностей ускорения разработки
клиентских приложений с использованием SASS}
\renewcommand{\mysubject}{Веб-технологии}
\renewcommand{\mylecturer}{Дрозин А.Ю.}

\begin{document}
\begin{titlepage}
    
    \thispagestyle{empty}
    
    \begin{center}
        
        Министерство науки и Высшего образования Российской Федерации \\
        Севастопольский государственный университет \\
        Кафедра ИС
        
        \vfill

        Отчет \\
        по лабораторной работе №\mylabnumber \\
        \enquote{\mylabtitle} \\
        по дисциплине \\
        \enquote{\MakeTextUppercase{\mysubject}}

    \end{center}

    \vspace{1cm}

    \noindent\hspace{7.5cm} Выполнил студент группы ИС/б-17-2-о \\
    \null\hspace{7.5cm} Горбенко К. Н. \\
    \null\hspace{7.5cm} Проверил \\
    \null\hspace{7.5cm} \mylecturer

    \vfill

    \begin{center}
        Севастополь \\
        \the\year{}
    \end{center}

\end{titlepage}

\section{Цель работы}
Изучить возможности метаязыка SASS для упрощения разработки
файлов каскадных таблиц стилей, приобрести практические навыки
использования SASS/SCSS при реализации веб-страниц.

\section{Задание на работу}
В процессе выполнения лабораторной работы необходимо реализовать
файл стилей заданного вариантом задания раздела сайта с использованием
SASS/SCSS. Необходимо использовать следующие возможности языка:
\begin{enumerate}
    \item вложения;
    \item расширения;
    \item заполнения;
    \item математика.
\end{enumerate}

\section{Ход работы}
\subsection{Миграция общих стилей}
Перепишем файлы с общими стилями \code{shared.css}, \code{nav.css}
\code{forms.css} на \code{.scss}. Получили следующие результаты:

Файл \code{shared.scss}.
\begin{lstlisting}
$defaultBorderRadius: 5px;

body {
    font-family: Arial, Helvetica, sans-serif;
    margin: 0;
    background-color: #f2f2f2;

    .portrait {
        max-width: 100%;
    }

    .main-content {
        position: relative;
        top: -100px;
        width: 60%;
        margin: 0 auto;
        padding: 50px 100px;
        background-color: white;
        border-radius: $defaultBorderRadius;
        box-shadow: 0 4px 8px 0 rgba(0, 0, 0, 0.2), 0 6px 20px 0 rgba(0, 0, 0, 0.19);
        overflow: hidden;

        .avatar {
            width: 30%;
            float: left;
            margin: 0 2% 2% 0;
        }

        table {
            tr {
                height: 60px;
                text-align: center;

                th {
                    font-weight: bold;
                    background-color: lightgrey;
                }

                td {
                    background-color: #e3e3e3;
                }
            }
        }

        .photo-wrapper {
            display: flex;
            flex-flow: row wrap;
            align-content: flex-start;
            justify-content: left;

            img {
                width: 30%;
                max-height: 300px;
                flex: 2 2 auto;
                margin: 10px;
            }
        }
    }

    footer {
        background-color: #e6e6e6;

        section {
            width: 60%;
            margin: 0 auto;
            padding: 50px 100px;
        }
    }
}

.full-width {
    width: 100%;
}

p {
    text-align: justify;
}

header * {
    font-weight: 100;
}

h1 {
    font-size: 40px;
}

h2 {
    font-size: 32px;
}

button {
    font-size: 20px;
    border: none;
    margin: 0 10px;
    width: 100px;
    height: 40px;
    border-radius: $defaultBorderRadius;
    background-color: lightgrey;

    &:hover {
        background-color: darken(lightgrey, 20%);
    }
}
\end{lstlisting}

Файл \code{nav.scss}:

\begin{lstlisting}
$navHeight: 60px;

nav {
    border-bottom: solid black 1px;
    height: $navHeight;

    ul {
        display: flex;
        list-style: none;
        flex-flow: row nowrap;
        justify-content: center;
        align-items: stretch;
        padding-left: 0;
        width: 70%;
        margin: 0 auto;
        height: 100%;

        li {
            flex: 3 3 auto;
            text-align: center;

            span {
                line-height: $navHeight;

                &#date, &#time {
                    line-height: $navHeight / 2;
                }
            }

            a {
                @extend span;

                text-decoration: none;
                text-transform: none;
                color: black;

                &:hover {
                    color: #bfbfbf;
                }
            }
        }
    }
}
\end{lstlisting}

Файл \code{forms.scss}:
\begin{lstlisting}
@import "shared";

$elementMargin: 5px;
$invalidInputColor: #cb2431;

form {
    ol {
        margin: 10px auto;
        max-width: 400px;
        font-weight: 100;

        li {
            display: block;
            list-style: none;
            padding: 0 5px;

            label {
                display: block;
                text-align: center;
                margin: $elementMargin 0;
                padding: 5px;
            }

            %form-item {
                border: solid 1px #dedede;
                border-radius: $defaultBorderRadius;
                outline: none;
                margin: $elementMargin 0;
                padding: 5px;

                &.error-messages {
                    color: $invalidInputColor;
                    margin-bottom: 5px;
                }

                &:focus {
                    box-shadow: 0 0 0 2px #308DE0;
                }

                &.validation-failed {
                    box-shadow: 0 0 0 2px $invalidInputColor;
                }
            }

            select {
                @extend %form-item;

                width: 100%;
            }

            textarea {
                @extend %form-item;

                height: 100px;
                resize: vertical;
            }

            input {
                @extend %form-item;

                &[type=submit] {
                    color: white;
                    background: #4B99AD;
                    border: none;
                    border-radius: $defaultBorderRadius;
                    padding: 7px 15px;

                    &:hover {
                        background: #4691A4;
                        box-shadow: none;
                    }
                }
            }

            .inline {
                display: inline-block;
            }

            .field-long {
                width: 100%;
            }
        }
    }
}
\end{lstlisting}

Теперь, когда общие файлы стилей готовы, необходимо включить их в бандлы.
Для этого в \code{.ts} файлах, являющихся входными точками для страниц
введем импорт соответствующих файлов стилей, выглядящих следующим образом:

\begin{lstlisting}
@import 'shared';
@import "nav";

...// Стили соответствующей страницы.
\end{lstlisting}

Импорт этих файлов выглядит следующим образом (\code{.ts} файл):

\begin{lstlisting}
import * as $ from 'jquery';

import { updateClockOnInterval } from '../clock/clock';
import { visitPage } from '../storage/storage';

import '../scss/index.scss';
\end{lstlisting}

Аналогичные действия проведем для всех страниц сайта.

\subsection{Миграция стилей компонентов}
Осуществим миграцию компонентов модального окна, меню переключения фотографий,
дейтпикера.

Файл \code{modal.scss}:

\begin{lstlisting}
$modalConfirmButtonColor: #4CAF50;
$modalCancelButtonColor: #f44336;

.modal {
    position: fixed;
    z-index: 1;
    left: 0;
    top: 0;
    width: 100%;
    height: 100%;
    background-color: rgba(0,0,0,0.4);
    display: none;

    .modal-window {
        position: absolute;
        top: 50%;
        left: 50%;
        transform: translate(-50%, -50%);
        max-width: 30%;
        max-height: 30%;
        padding: 20px;
        background-color: lightgrey;
        border-radius: 5px;

        .modal-message {
            font-size: 28px;
            text-align: center;
        }

        .modal-navigation {
            text-align: center;
            margin: 20px;

            .modal-button {
                width: 100px;
                height: 40px;
                border-radius: 5px;
                font-size: 20px;
                border: none;
                margin: 0 10px;

                &.modal-confirm-button {
                    background-color: $modalConfirmButtonColor;

                    &:hover {
                        background-color: darken($modalConfirmButtonColor, 20%);
                    }
                }

                &.modal-cancel-button {
                    background-color:  $modalCancelButtonColor;

                    &:hover {
                        background-color: darken($modalCancelButtonColor, 20%);
                    }
                }
            }
        }
    }
}

.no-scroll {
    overflow: hidden;
}
\end{lstlisting}

Файл \code{lightbox.scss}:

\begin{lstlisting}
#lightbox {
    position: fixed;
    z-index: 1;
    left: 0;
    top: 0;
    width: 100%;
    height: 100%;
    background-color: rgba(0,0,0,0.4);
    display: none;

    .lightbox-content {
        position: absolute;
        top: 50%;
        left: 50%;
        transform: translate(-50%, -50%);
        height: 80%;

        .image-wrapper {
            display: flex;
            max-width: 100%;
            max-height: 100%;

            img {
                max-width: 100%;
                max-height: 100%;
                flex: none;
            }
        }

        .lightbox-button {
            font-size: 60px;
            line-height: 40px;
            color: white;
            cursor: pointer;

            &.lightbox-arrow-left {
                float: left;
            }

            &.lightbox-arrow-right {
                float: right;
            }

            &:hover {
                text-shadow: 0 0 10px white;
            }
        }
    }

    .lightbox-close {
        @extend .lightbox-button;

        position: fixed;
        top: 5%;
        right: 5%;
    }
}

.no-scroll {
    overflow: hidden;
}
\end{lstlisting}

Файл \code{datepicker.scss}:

\begin{lstlisting}
#datepicker {
    display: block;
    width: 100%;
    margin-top: 10px;
    padding: 5px 0;
    border-radius: 4px;
    background-color: #dddddd;

    .element-wrapper {
        width: 100%;
        margin: 10px 0;

        #datepicker-year, #datepicker-month {
            display: inline-block;
            width: 40%;
            margin: 0 5%;
        }

        #datepicker-date-list {
            width: 90%;
            list-style: none;
            padding-left: 0;
            margin-left: 5%;
            margin-right: 5%;
            display: flex;
            flex-flow: row wrap;

            li {
                flex: 0 1 20%;
                text-align: center;
                justify-content: left;
                margin: 0;
                height: 40px;
                line-height: 40px;

                &:hover {
                    background-color: #cccccc;
                    cursor: pointer;
                }
            }
        }
    }
}
\end{lstlisting}

Подключим полученные файлы к бандлам соответствующих страниц:

\begin{lstlisting}
import '../scss/contact.scss';
import '../datepicker/datepicker.scss';
\end{lstlisting}

\section*{Выводы}
В ходе лабораторной работы была осуществлена миграция стилей сайта на
метаязык \code{scss}. По сравнению с \code{css}, \code{scss} позволяет
повторно использовать одни стили для нескольких селекторов, что значительно
упрощает структуру документа. Кроме того, он развешает вложенные селекторы,
объявления функций (миксины) и арифметические вычисления значений свойств.
\end{document}