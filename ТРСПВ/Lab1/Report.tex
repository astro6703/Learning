\documentclass[a4paper,14pt]{extarticle}
\usepackage{../../tex-shared/report-layout}

\renewcommand{\mylabnumber}{1}
\renewcommand{\mylabtitle}{Исследование средств создания распределено выполняющихся программ}
\renewcommand{\mysubject}{Теория распределенных систем и параллельных вычислений}
\renewcommand{\mylecturer}{Дрозин А. Ю.}

\begin{document}
\begin{titlepage}
    
    \thispagestyle{empty}
    
    \begin{center}
        
        Министерство науки и Высшего образования Российской Федерации \\
        Севастопольский государственный университет \\
        Кафедра ИС
        
        \vfill

        Отчет \\
        по лабораторной работе №\mylabnumber \\
        \enquote{\mylabtitle} \\
        по дисциплине \\
        \enquote{\MakeTextUppercase{\mysubject}}

    \end{center}

    \vspace{1cm}

    \noindent\hspace{7.5cm} Выполнил студент группы ИС/б-17-2-о \\
    \null\hspace{7.5cm} Горбенко К. Н. \\
    \null\hspace{7.5cm} Проверил \\
    \null\hspace{7.5cm} \mylecturer

    \vfill

    \begin{center}
        Севастополь \\
        \the\year{}
    \end{center}

\end{titlepage}

\section{Цель работы}
Исследовать функции библиотеки MPI, необходимые для создания и взаимодействия
распределено выполняемых программ.

\section{Постановка задачи}
Вариант №1. Программа осуществляет умножение двух матриц. Размеры матриц -- $3 * 3$ и
$4 * 4$. На каждом процессе, определяет произведение одной строки первой матрицы на
все столбцы второй матрицы. Результаты возвращаются в родительскую задачу.

\section{Ход работы}
Исходный код программы представлен на листинге:
\begin{lstlisting}
#include <iostream>
#include <fstream>
#include <mpi.h>

using namespace std;

MPI_Status status;

void master_send(int **m_a, int **m_b, int n, int cor, int i, int j, int comm) {
    int buf[100];
    buf[0] = n;
    buf[1] = cor;
    for (int l = 0; l < n; l++) {
        buf[2 + l] = m_a[i][l];
    }
    for (int l = 0; l < n; l++) {
        buf[2 + n + l] = m_b[l][j];
    }
    MPI_Send(buf, 2 * n + 2, MPI_INT, comm, 1, MPI_COMM_WORLD);
}

void master() {
    int size;
    MPI_Comm_size(MPI_COMM_WORLD, &size);
    if (size == 1) {
        cout << "Can`t run without slaves! Just buy some..." << endl;
        return;
    }
    ifstream is("input.txt");

    int n, m, k;
    is >> n >> m >> k;

    int **m_a = new int *[n];
    int **m_b = new int *[m];
    for (int i = 0; i < n; i++) {
        m_a[i] = new int[m];
        for (int j = 0; j < m; j++) {
            is >> m_a[i][j];
        }
    }
    for (int i = 0; i < m; i++) {
        m_b[i] = new int[k];
        for (int j = 0; j < k; j++) {
            is >> m_b[i][j];
        }
    }
    is.close();


    bool used[size];
    memset(used, 0, sizeof(used));
    used[0] = true;
    long long m_c[n][k];
    int len = n * k;
    int j = 0;
    int online = 0;
    for (int i = 1; i < size && j < len; i++) {
        master_send(m_a, m_b, m, j, j / k, j % k, i);
        used[i] = true;
        online++;
        j++;
    } 

    long message[2];
    while (j < len || online > 0) {
        MPI_Recv(message, 2, MPI_LONG_LONG, MPI_ANY_SOURCE, 1, MPI_COMM_WORLD, &status);
        m_c[message[0] / k][message[0] % k] = message[1];
        used[status.MPI_SOURCE] = false;
        online--;
  
        if (j < len) {
            master_send(m_a, m_b, m, j, j / k, j % k, status.MPI_SOURCE);
            used[status.MPI_SOURCE] = true;
            online++;
            j++;
        } 
    } 
  
    for (int i = 1; i < size; i++) {
        int buf[1];
        MPI_Send(buf, 0, MPI_LONG_LONG, i, 2, MPI_COMM_WORLD);
    }

    for (int i = 0; i < n; i++) {
        for (int j = 0; j < k; j++) {
            cout << m_c[i][j] << " ";
        }
        cout << endl;
    }
}

void slave() {
    int message[100];
    bool running = true;
    while (running) {
        MPI_Recv(message, 100, MPI_INT, 0, MPI_ANY_TAG, MPI_COMM_WORLD, &status);
        if (status.MPI_TAG == 2) {
            running = false;
        } else {
            int n = message[0];
            long long result[2] = {message[1], 0};
            for (int i = 0; i < n; i++) {
                result[1] += message[2 + i] * message[2 + n + i];
            }
            MPI_Send(result, 2, MPI_LONG_LONG, 0, 1, MPI_COMM_WORLD);
        }
    }
}

int main(int argc, char **argv) {
    int rank;

    MPI_Init(&argc, &argv);
    MPI_Comm_rank(MPI_COMM_WORLD, &rank);

    rank ? slave() : master();

    MPI_Barrier(MPI_COMM_WORLD);
    MPI_Finalize();
    return 0;
}
\end{lstlisting}

\section*{Вывод}
В ходе данной лабораторной работы были изучены основные принципы библиотеки MPI
и ее функции. Написана программа, осуществляющая распределённые вычисления на
любом количества хостов (не менее 2х), в которой один из хостов -- главный, а
все остальные -- подчиненные. Главный процесс распределяет задачи по умножению
между подчиненными, которые осуществляют умножение. После выполнения задачи всем
подчиненным процессам отправляется пустое сообщение с определённым тегом для
завершения их ожидания.

Отмечу, что использование количества процессов, равное произведению количества
строк первой матрицы на количество столбцов второй + 1 является наиболее
эффективным, так как все задачи распределяются одновременно и главный процесс не
вынужден ожидать момента, когда один из процессов освободится для распределения
следующей части задания. При этом большее количество процессов будет избыточным,
так как каждый последующий процесс в итоге так и не будет задействован.

\end{document}