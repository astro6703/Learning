\documentclass[a4paper,14pt]{extarticle}
\usepackage{../../tex-shared/report-layout}

\renewcommand{\mylabnumber}{6}
\renewcommand{\mylabtitle}{Создание динамических баз данных}
\renewcommand{\mysubject}{Методы и средства искусственного интеллекта}
\renewcommand{\mylecturer}{Забаштанский А.К.}

\begin{document}
\begin{titlepage}
    
    \thispagestyle{empty}
    
    \begin{center}
        
        Министерство науки и Высшего образования Российской Федерации \\
        Севастопольский государственный университет \\
        Кафедра ИС
        
        \vfill

        Отчет \\
        по лабораторной работе №\mylabnumber \\
        \enquote{\mylabtitle} \\
        по дисциплине \\
        \enquote{\MakeTextUppercase{\mysubject}}

    \end{center}

    \vspace{1cm}

    \noindent\hspace{7.5cm} Выполнил студент группы ИС/б-17-2-о \\
    \null\hspace{7.5cm} Горбенко К. Н. \\
    \null\hspace{7.5cm} Проверил \\
    \null\hspace{7.5cm} \mylecturer

    \vfill

    \begin{center}
        Севастополь \\
        \the\year{}
    \end{center}

\end{titlepage}

\section{Цель работы}
Изучение технологии подготовки и выполнения Пролог-программ в интегрированной
среде, исследование способов организации динамических баз данных (БД) средствами
языка Пролог.

\section{Постановка задачи}
Корректировка данных в базе по номеру поезда; вывод на экран информации о
поездах, отправляющихся после введенного с клавиатуры времени; если таких
поездов нет -- выдать на дисплей соответствующее сообщение.

\section{Ход работы}
Текст программы:
\begin{lstlisting}
:-dynamic %информирует интерпретатор о том, что определения предикатов 
%могут изменяться в ходе выполнения программы 
время/2, 
поезд/3, % формат: <имя предиката>/<кол-во аргументов> 
сотрудник_ф/8. 
% первоначальная база, загружаемая при запуске программы 
% Пункт назначения, № поезда, врем отправления. 
поезд('Москва', 101, время(14, 00)). 
поезд('Казань', 102, время(12, 00)). 
поезд('Казань', 103, время(13, 00)). 
поезд('Москва', 103, время(23, 00)). 
поезд('Симферополь', 104, время(11, 30)). 
start:- menu. %предикат для запуска программы 
  
%0============= отображение меню ============================================== 
menu:- 
repeat, nl, 
write('*******************************'),nl, 
write('* 1. Добавление записи в БД *'),nl, 
write('* 2. Удаление записи из БД *'),nl, 
write('* 3. Выборка записей из БД *'),nl, 
write('* 4. Просмотр БД *'),nl, 
write('* 5. Загрузка БД из файла *'),nl, 
write('* 6. Сохранение БД в файле *'),nl, 
write('* 7. Реляционные операции *'),nl, 
write('* 8. Выход *'),nl, 
write('*******************************'), nl ,nl, 
write('Введите номер пункта меню с точкой в конце!!!'),nl, 
read(C),nl, %Ввод пункта меню 
proc(C), %Запуск процедуры с номером С 
C=8, %Если С не равно 8, то авт. возврат к repeat 
!. %Иначе успешное завершение 
%0----------------------------------------------------------------------------- 
  
%1======= добавление записи в базу данных ===================================== 
proc(1):- 
write('Ввод завершайте точкой!!! :'),nl, 
write('Введите пункт назначения:'),nl, read(Пункт), 
write('Введите номер поезда:'),nl, read(N), 
write('Введите час отправления:'),nl, read(Час), 
write('Введите минуту отправления:'),nl, read(Мин), 
assertz(поезд(Пункт,N,время(Час,Мин))), %добавление факта в БД 
write('Поезд №'),write(N),write(' был добавлен в БД'),nl, 
write('Введите любой символ'),nl, %ожидание ввода литеры 
get0(_). 
%1----------------------------------------------------------------------------- 
  
%2========= удаление записи из базы данных ==================================== 
proc(2):- 
write('Введите номер поезда для удаления'), nl, 
read(N), 
retract(поезд(_,N,_)), %удаление записи 
write('Поезд №'), 
write(N), tab(2), %вывод сообщения об успешном удалении 
write('был успешно удален из БД'),nl, 
write('Введите любой символ'),nl, 
get0(_), %ожидание ввода символа 
!; %отсечение альтернативы и завершение 
write('Такого поезда:'),tab(2), %вывод сообщения о безуспешном удалении 
write('в базе данных нет'),nl, 
write('Введите любой символ'),nl, 
get0(_). %ожидание ввода символа 
%2----------------------------------------------------------------------------- 
  
%3====== выборка записи из базы данных по критерию ============================ 
%------- выбираются поезда, отправляющиеся после времени ------------------------- 
proc(3):- 
write('Введите час отправления:'), nl, 
read(Ч2), %ввод часа 
write('Введите час отправления:'), nl, 
read(М2), %ввод минуты 
retractall(flag(_)), %удаление фактов - flag(_) 
nl,write('Поезда:'),nl, 
поезд(Пункт,N,Время), %выбор записи о сотруднике 
Время = время(Ч1,М1), 
отправляется_после(Время,время(Ч2,М2)), %проверка критерия 
assert(flag(1)), %запомнить флаг – запись найдена 
write('Номер: '), write(N),nl, 
write('Пункт назачения: '), write(Пункт), nl, 
write('Время отправлеия: '), write(Ч1), write(':'), write(М2), nl,nl, 
fail; %возврат для выбора след. записи 
flag(1),!. %eсли записи были найдены, то завершить успешно 
  
proc(3):- %cообщение, если записи не найдены 
write('В базе нет таких поездов'),nl, 
write('Введите любой символ'),nl, 
get0(_),get0(_). 
  
%проверка времени после 
отправляется_после(Время,После):- 
Время = время(Ч1,М1), 
После = время(Ч2,М2), 
время_больше(Ч1,М1,Ч2,М2). 
  
время_больше(Ч1,М1,Ч2,М2):- 
Ч1>Ч2; 
Ч1=:=Ч2, 
М1>=М2. 
%3----------------------------------------------------------------------------- 
  
%4================== просмотр базы данных ===================================== 
proc(4):- 
write('Поезда:'),nl, 
поезд(Пункт,N,Время), %извлечение записи из БД 
Время = время(Час,Мин), 
write('Номер поезда: '), write(N),nl, %отображение на дисплее 
write('Пункт назначения: '), write(Пункт), nl, %элементов запаси 
write('Время: '), write(Час), write(':'), write(Мин),nl,nl, 
fail; %возврат к выбору записи 
write('Введите любой символ'),nl, 
get0(_),get0(_), %ожидание ввода символ 
true. %завершение - записей больше нет 
%4----------------------------------------------------------------------------- 
  
%5======== загрузка базы данных из файла ====================================== 
proc(5):- 
see('lab1.dat'), %текущий входной поток - lab1.dat 
retractall(поезд(_,_,_)),%очистка БД от фактов "поезд" 
db_load, %загрузка БД термами из файла 
seen, %закрытие потока 
write('БД загружена из файла'),nl. 
%загрузка термов в БД из открытого вх. потока 
db_load:- 
read(Term), %чтение терма 
(Term == end_of_file,!; %если конец файла, то завершение 
assertz(Term), %иначе добавить терм в конец БД 
db_load). %рекурсивный вызов для чтения след. терма 
%5----------------------------------------------------------------------------- 
  
%6========== сохранение БД в файле ============================================ 
proc(6):- 
tell('lab1.dat'), %открытие вых. потока 
save_db(поезд(Пункт,N,Время)), %сохранение терма 
told, %закрытие вых. потока 
write('БД скопирована в файл lab1.dat'),nl. 
%сохранение терма в открытом файле 
save_db(Term):- %сохранение терма (факта!) Term в БД 
Term, %отождествление терма с термом в БД 
write_term(Term, [quoted(true)]), %запись терма 
write('.'),nl, %запись точки в конце терма 
fail; %неудача с целью поиска след. варианта 
true. %завершение, если вариантов отождествления нет 
%6----------------------------------------------------------------------------- 
  
%7============ реализация операций реляционной алгебры ======================== 
proc(7):- 
write('Формирование отношения r1: поезда в Москву'), nl, 
подмножество_поездов('Москва',R1), %R1 - список поездов в Москву 
список_в_бд(R1), %добавление элементов из R1 в базу данных 
вывод_списка(R1),nl, %вывод списка R1 на экран 
write('Формирование отношения r2: поезда в Казань'), nl, 
подмножество_поездов('Казань',R2), %R2 - список поездов в Казань 
список_в_бд(R2), %добавление элементов из R2 в базу данных 
вывод_списка(R2),nl, %вывод списка R2 на экран 
write('Объединенное отношение r1_или_r2: '), nl, 
объединение('Москва','Казань',Rez1), %Rez1 - список поездов в Москву и Казань 
вывод_списка(Rez1),nl, 
write('Пересечение отношений r1_и_r2: '), nl, 
пересечение('Москва','Казань',Rez2), %Rez2 - список поездов в оба города 
вывод_списка(Rez2),nl, 
write('Разность отношений r1-r2: '), nl, 
разность('Москва','Казань',Rez3), %Rez3-список поездов в Москву, но не в Казань 
вывод_списка(Rez3),nl, 
write('Введите любой символ'),nl, 
get0(_),get0(_). 
%------------------------------------------------------------------------------ 
  
подмножество_поездов(Пункт,R):- 
bagof(поезд_п(Пункт,N,Время), 
поезд(Пункт,N,Время), R). 
  
объединение_r1_r2(П1,П2,Пункт,N,Время):- 
поезд_п(П1,N,Время),Пункт=П1; 
поезд_п(П2,N,Время),Пункт=П2. 
  
объединение(П1,П2,Rez):- 
bagof(поезд_п1_или_п2(Пункт,N,Время), 
объединение_r1_r2(П1,П2,Пункт,N,Время), %условие вкл. в список 
Rez). 
  
пересечение_r1_r2(П1,П2,N,В1,В2):- 
поезд_п(П1,N,В1), 
поезд_п(П2,N,В2). 
  
пересечение(П1,П2,Rez):- 
bagof(поезд_п1_и_п2(П1,N,В1,П2,N,В2), 
пересечение_r1_r2(П1,П2,N,В1,В2), 
Rez). 
  
разность_r1_r2(П1,П2,N,В1,В2):- 
              поезд_п(П1,N,В1), 
              not(поезд_п(П2,N,В2)). 
  
разность(П1,П2,Rez):- 
bagof(поезд_п1_и_не_п2(П1,N,В1), 
разность_r1_r2(П1,П2,N,В1,В2), 
Rez). 
  
  
список_в_бд([]). 
список_в_бд([H|T]):- 
H=поезд_п(Пункт,N,Время), 
assertz(поезд_п(Пункт,N,Время)), 
список_в_бд(T). 
  
вывод_списка([]). 
вывод_списка([H|T]):-write(H),nl,вывод_списка(T). 
%7----------------------------------------------------------------------------- 
%8============выход============================================================ 
proc(8):-write('Ну и ладно...'),nl. 
%8-----------------------------------------------------------------------------  
\end{lstlisting}

\section*{Выводы}
В ходе работы были изучены технологии подготовки и выполнения Пролог-программ в
интегрированной среде, исследование способов организации динамических баз данных
(БД) средствами языка Пролог.

\end{document}