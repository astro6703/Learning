\documentclass[a4paper,14pt]{extarticle}
\usepackage{../../tex-shared/preamble}

\renewcommand{\mylabnumber}{7}
\renewcommand{\mylabtitle}{Исследование возможностей ускорения
разработки клиентских приложений с использованием CoffeeScript}
\renewcommand{\mysubject}{Веб-технологии}
\renewcommand{\mylecturer}{Дрозин А.Ю.}

\begin{document}
\begin{titlepage}
    
    \thispagestyle{empty}
    
    \begin{center}
        
        Министерство науки и Высшего образования Российской Федерации \\
        Севастопольский государственный университет \\
        Кафедра ИС
        
        \vfill

        Отчет \\
        по лабораторной работе №\mylabnumber \\
        \enquote{\mylabtitle} \\
        по дисциплине \\
        \enquote{\MakeTextUppercase{\mysubject}}

    \end{center}

    \vspace{1cm}

    \noindent\hspace{7.5cm} Выполнил студент группы ИС/б-17-2-о \\
    \null\hspace{7.5cm} Горбенко К. Н. \\
    \null\hspace{7.5cm} Проверил \\
    \null\hspace{7.5cm} \mylecturer

    \vfill

    \begin{center}
        Севастополь \\
        \the\year{}
    \end{center}

\end{titlepage}

\section{Цель работы}
Изучить основные возможности языка CoffeeScript, приобрести практические
навыки написания клиентских приложений с использованием CoffeeScript.

\section{Задание на работу}
В процессе выполнения лабораторной работы необходимо переработать клиентский
код страниц «Контакт», «Тест по дисциплине», «Фотоальбом» и «Мои интересы»,
реализовав его на языке CoffeeScript.

\section{Ход работы}
Для парсинга \code{.coffee} скриптов подключим соответствующий лоадер:

\code{npm install --save-dev coffeescript coffee-loader}.

Теперь перепишем код файлов \code{interests.ts}, \code{photos.ts},
\code{contact.ts}, \code{test.ts}.

Файл \code{interests.coffee}:

\begin{lstlisting}
import * as $ from 'jquery';

import { updateClockOnInterval } from '../clock/clock';
import { visitPage } from '../storage/storage';
import modal from '../modal/modal';

import '../scss/interests.scss';
import '../modal/modal.scss';

$ () ->
    visitPage 'interests'
    updateClockOnInterval (document.getElementById('date')), (document.getElementById 'time'), 1000
    $('.contents').on('click', (event) -> $(event.target).next().toggle())
    $('#dialog').on('click', () -> modal({ confirmAction: () -> alert 'lol' }))
\end{lstlisting}

Подключим его следующим образом (\code{webpack.config.js}):
\begin{lstlisting}
entry: {
    index: './src/pages/index.ts',
    about: './src/pages/about.ts',
    interests: './src/pages/interests.ts',
    learning: './src/pages/learning.ts',
    photos: './src/pages/photos.ts',
    contact: './src/pages/contact.ts',
    test: './src/pages/test.ts',
    history: './src/pages/history.ts',
    interestsCoffee: './src/coffee/interests.coffee',
}
\end{lstlisting}

\code{interests.haml}:

\code{\%script\{ src: "../js/interestsCoffee.js" \}}.
Следующие файлы подключаются аналогично.

Файл \code{photos.coffee}:
\begin{lstlisting}
import * as $ from 'jquery';

import { updateClockOnInterval } from '../clock/clock';
import { createElement } from '../utils/dom';
import addLightbox from '../lightbox/lightbox';
import { visitPage } from '../storage/storage';

import '../scss/photos.scss';
import '../lightbox/lightbox.scss';

$ () ->
    visitPage 'photos'
    updateClockOnInterval (document.getElementById 'date'), (document.getElementById 'time'), 1000
    photoWrapper = document.querySelector '.photo-wrapper'

    photos.forEach (photo) ->
        img = createElement 'img', { src: photo, alt: 'photoalbum photo' }
        photoWrapper.appendChild(img)

    addLightbox photoWrapper, photos

photos = [
    'picachu.jpg',
    'images.jpeg',
    'lion-king.jpg',
    'road.jpeg',
    'car.png',
    'bridge.jpg',
    'green.jpeg',
    'statue.jpg',
    'cock.jpg',
    'city.jpg',
    'fire.jpg',
    'rose.jpeg',
    'photo.jpg',
    'castle.jpg',
    'shimpanze.jpg',
].map (image) -> "../images/#{image}"
\end{lstlisting}

Файл \code{contact.coffee}:
\begin{lstlisting}
import * as $ from 'jquery';

import { updateClockOnInterval } from '../clock/clock';
import { FormComponent, NameValidator, FieldFilledValidator, PhoneNumberValidator, setFieldsForValidation, DateValidator } from '../forms/forms';
import datepicker from '../datepicker/datepicker';
import { visitPage } from '../storage/storage';

import '../scss/contact.scss';
import '../datepicker/datepicker.scss';
import '../../node_modules/tooltipster/dist/css/tooltipster.bundle.min.css';
import '../../node_modules/tooltipster/dist/css/plugins/tooltipster/sideTip/themes/tooltipster-sideTip-light.min.css';

$ () ->
    visitPage 'contact'
    updateClockOnInterval (document.getElementById 'date'), (document.getElementById 'time'), 1000
    datepicker()
    setFieldsForValidation fields, document.forms.item 0

fields = [
    new FormComponent(
        'full-name-input',
        [ new NameValidator ],
        'Type three space-separated words.'
    ),
    new FormComponent(
        'email-input',
        [ new FieldFilledValidator ],
        'Email field should contain "@" character.'
    ),
    new FormComponent(
        'phone-input',
        [ new PhoneNumberValidator ],
        'Type phone number in next format: +XXXXXXXXXX.'
    ),
    new FormComponent(
        'message-input',
        [ new FieldFilledValidator ],
        'Message field should be filled.'
    ),
    new FormComponent(
        'datepicker-input',
        [ new DateValidator ],
        'Click on field to choose date.'
    )
]
\end{lstlisting}

Файл \code{test.coffee}:
\begin{lstlisting}
import * as $ from 'jquery';

import {updateClockOnInterval} from '../clock/clock';
import {
    FormComponent,
    NameValidator,
    FieldFilledValidator,
    PhoneNumberValidator,
    DetailedAnswerValidator,
    setFieldsForValidation
} from '../forms/forms';
import {visitPage} from '../storage/storage';

import '../scss/test.scss';
import '../datepicker/datepicker.scss';
import '../../node_modules/tooltipster/dist/css/tooltipster.bundle.min.css';
import '../../node_modules/tooltipster/dist/css/plugins/tooltipster/sideTip/themes/tooltipster-sideTip-light.min.css';

$ () ->
    visitPage 'test'
    updateClockOnInterval (document.getElementById 'date'), (document.getElementById 'time'), 1000
    setFieldsForValidation fields, document.forms.item 0

fields = [
    new FormComponent(
        'question3',
        [ new FieldFilledValidator, new DetailedAnswerValidator ],
        'Field should be filled and should contain less than 30 words.'
    ),
    new FormComponent(
        'full-name-input',
        [ new NameValidator ],
        'Type three space-separated words.'
    ),
    new FormComponent(
        'email-input',
        [ new FieldFilledValidator ],
        'Email field should contain "@" character.'
    ),
    new FormComponent(
        'phone-input',
        [ new PhoneNumberValidator ],
        'Type phone number in next format: +XXXXXXXXXX.'
    ),
    new FormComponent(
        'message-input',
        [ new FieldFilledValidator ],
        'Message field should be filled.'
    )
]
\end{lstlisting}

\section*{Выводы}
В ходе лабораторной работы для разработки клиентского кода был использован
язык \code{CoffeeScript}. До выхода es6 он предоставлял более простой синтаксис,
а именно: объявления функций, похожие на современные анонимные функции es6,
объявления переменных с локальной областью видимости, отказ от большинства
общепринятых скобок (при передаче параметров в функции, при объявлении
объектов), итерации по массивам.

Эти функции языка, несомненно, были преимуществом перед более старыми
версиями языка \code{JavaScript}. Но \code{JS} в настоящее время развивается более активно,
и, с развитием, сам обзавелся вышеупомянутыми функциями или их аналогами:
стрелочные функции, \code{let} и \code{const} объявления, циклы \code{for...of},
\code{for...in}, \code{forEach}.

В настоящее время я не вижу преимуществ \code{CoffeeScript}, из-за которых
я бы отдал ему предпочтение над \code{JavaScript}.
\end{document}