\documentclass[a4paper,14pt]{extarticle}
\usepackage{../../tex-shared/preamble}

\renewcommand{\mylabnumber}{2}
\renewcommand{\mylabtitle}{Исследование
возможностей программирования
на стороне клиента. Основы языка JavaScript}
\renewcommand{\mysubject}{Веб-технологии}
\renewcommand{\mylecturer}{Дрозин А.Ю.}

\begin{document}
\begin{titlepage}
    
    \thispagestyle{empty}
    
    \begin{center}
        
        Министерство науки и Высшего образования Российской Федерации \\
        Севастопольский государственный университет \\
        Кафедра ИС
        
        \vfill

        Отчет \\
        по лабораторной работе №\mylabnumber \\
        \enquote{\mylabtitle} \\
        по дисциплине \\
        \enquote{\MakeTextUppercase{\mysubject}}

    \end{center}

    \vspace{1cm}

    \noindent\hspace{7.5cm} Выполнил студент группы ИС/б-17-2-о \\
    \null\hspace{7.5cm} Горбенко К. Н. \\
    \null\hspace{7.5cm} Проверил \\
    \null\hspace{7.5cm} \mylecturer

    \vfill

    \begin{center}
        Севастополь \\
        \the\year{}
    \end{center}

\end{titlepage}

\section{Цель работы}
Изучить основы языка JavaScript и объектной модели браузера,
приобрести практические навыки проверки HTML-форм с использованием
JavaScript.

\section{Задание на работу}
\begin{enumerate}
    \item Модифицировать страницу «Фотоальбом» (использовать
          HTMLстраницы, разработанные при выполнении предыдущей
          лабораторной работы), реализовав вывод таблицы,
          содержащей фото, с использованием операторов циклов.
          Значения имен файлов фото и подписей к фото
          предварительно разместить в массивах fotos и titles.
    \item Модифицировать страницу «Мои интересы», реализовав вывод
          списков с использованием JavaScript-функции с переменным
          числом аргументов.
    \item Добавить на страницах «Контакт» и «Тест по дисциплине «…»»
          функции проверки заполненности форм. В случае если
          какое-либо из полей формы осталось незаполненным при нажатии
          на кнопку отправить, вывести сообщение об ошибке и
          установить фокус на незаполненный элемент.
    \item Добавить на странице «Контакт» текстовое поле «Телефон».
          И для полей «Фамилия Имя Отчество» и «Телефон» добавить
          функции специфической проверки значений. В случае если
          какое-либо из полей формы заполнено не верно, при нажатии
          на кнопку отправить, вывести сообщение об ошибке и установить
          фокус на неверно заполненный элемент. Формат правильных
          значений полей:
          
          \begin{itemize}
              \item Фамилия Имя Отчество – введено три слова,
                    разделенные одним пробелом.
              \item Телефон – строка может состоять только из цифр;
                    начинаться только с последовательности «+7» или «+3»;
                    не содержит пробелов; количество цифр в строке от
                    9 до 11.
          \end{itemize}
    \item Добавить на странице «Тест по дисциплине «…»» функции
          специфической проверки значений полей в соответствии
          с вариантом задания. В случае не выполнения условия
          сформировать сообщение об ошибке и установить фокус
          на неверно заполненный элемент ввода.
    \item Необходимо выполнить проверку разработанных JavaScript
          файлов с использованием сервиса jshint.
\end{enumerate}
\textbf{Вариант № 3}: 3 вопрос, количество слов в ответе не более 30.

\section{Ход работы}
\subsection{Страница \enquote{Фотоальбом}}
Для выведения фотографий средствами языка JavaScript была написана\
следующая программа:
\begin{lstlisting}
window.onload = () => {
    let photoWrapper = document.getElementsByClassName('photo-wrapper').item(0);
    photos.forEach(photo => {
        let img = document.createElement('img');
        img.setAttribute('src', photo);
        img.setAttribute('alt', 'photoalbum photo');
        photoWrapper.appendChild(img);
    });
};
const photos = [
    'picachu.jpg',
    'images.jpeg',
    'lion-king.jpg',
    'road.jpeg',
    'car.png',
    'bridge.jpg',
    'green.jpeg',
    'statue.jpg',
    'cock.jpg',
    'city.jpg',
    'fire.jpg',
    'rose.jpeg',
    'photo.jpg',
    'castrle.jpg',
    'shimpanze.jpg',
].map(image => image = `../images/${image}`);
\end{lstlisting}

В данной программе пути к фото хранятся в массиве \code{photos}.
При загрузке страницы происходит вызов функции, которая в цикле
добавляет фото в документ.
\end{document}