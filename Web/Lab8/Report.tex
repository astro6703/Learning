\documentclass[a4paper,14pt]{extarticle}
\usepackage{../../tex-shared/report-layout}

\renewcommand{\mylabnumber}{8}
\renewcommand{\mylabtitle}{Исследование архитектуры MVC приложения и
возможностей обработки данных HTML-форм на стороне сервера с использованием
языка PHP}
\renewcommand{\mysubject}{Веб-технологии}
\renewcommand{\mylecturer}{Овчинников А.Л.}

\begin{document}
\begin{titlepage}
    
    \thispagestyle{empty}
    
    \begin{center}
        
        Министерство науки и Высшего образования Российской Федерации \\
        Севастопольский государственный университет \\
        Кафедра ИС
        
        \vfill

        Отчет \\
        по лабораторной работе №\mylabnumber \\
        \enquote{\mylabtitle} \\
        по дисциплине \\
        \enquote{\MakeTextUppercase{\mysubject}}

    \end{center}

    \vspace{1cm}

    \noindent\hspace{7.5cm} Выполнил студент группы ИС/б-17-2-о \\
    \null\hspace{7.5cm} Горбенко К. Н. \\
    \null\hspace{7.5cm} Проверил \\
    \null\hspace{7.5cm} \mylecturer

    \vfill

    \begin{center}
        Севастополь \\
        \the\year{}
    \end{center}

\end{titlepage}

\section{Цель работы}
Изучить  основы синтаксиса PHP и принципы функционирования MVC приложения на
стороне сервера. Приобрести практические навыки использования языка PHP для
генерации HTML-кода и обработки HTML-форм в MVC приложении.

\section{Ход работы}
\subsection{Реализация маршрутизации}

Для реализации MVC архитектуры создадим маршрутизацию вида\\
\code{site/controller/action}, для этого используем \code{UrlRewrite} модуль и
создадим входную точку приложения index.php.

.htaccess:
\begin{lstlisting}
RewriteEngine On
RewriteCond %{REQUEST_FILENAME} !-f
RewriteCond %{REQUEST_FILENAME} !-d
RewriteRule ^(.*)/(.*) index.php?controller=$1&action=$2 [QSA]
\end{lstlisting}

index.php:
\begin{lstlisting}
<?php
include "Core/Router.php";

$request = $_REQUEST;
$requestMethod = $_SERVER["REQUEST_METHOD"];
Router::route($request, $requestMethod);
\end{lstlisting}

Теперь реализуем роутер, с разной мршрутизацией для \code{GET} и \code{POST}
запросов.

Core/Router.php
\begin{lstlisting}
class Router
{
    protected static $actionPrefixesByMethod = array(
        "GET" => "",
        "POST" => "Post"
    );

    static function route($request, $requestMethod) {
        $controllerName = self::getControllerName($request);
        $actionName = self::getActionName($request, $requestMethod);

        if ($requestMethod != "GET" && $requestMethod != "POST") {
            http_response_code(ResponseCodes::$notAllowedMethodStatusCode);
            die();
        }

        $controllerFileName = "Controllers/" . $controllerName . 'Controller.php';
        if (file_exists($controllerFileName)) {
            include $controllerFileName;
        } else {
            http_response_code(ResponseCodes::$notFoundStatusCode);
            die();
        }

        $controllerClassName = ucfirst($controllerName) . 'Controller';
        $container = Container::getInstance();
        $controller = new $controllerClassName($container);

        if (method_exists($controller, $actionName)) {
            $controller->$actionName();
        } else {
            http_response_code(ResponseCodes::$notFoundStatusCode);
            die();
        }
    }

    private static function getControllerName($request) {
        return array_key_exists("controller", $request)
            ? $request["controller"]
            : "Home";
    }

    private static function getActionName($request, $method) {
        if (!array_key_exists("action", $request)) {
            return "Index" . self::$actionPrefixesByMethod[$method];
        }

        return $request["action"] . self::$actionPrefixesByMethod[$method];
    }
}
\end{lstlisting}

Данный класс, учитывая имена контроллера и экшена из строки параметров,
создает экземпляр соответствующего контроллера и вызывает соответствующий экшен,
если они существуют. Иначе, он возвращает 404 ответ.

\subsection{Реализация MVC}
Архитектура MVC состоит из контроллеров, моделей и представлений. Создадим
первый контроллер:

Controllers/HomeContoller.php:
\begin{lstlisting}
class HomeController
{
    private $photosRepository;

    public function __construct(IContainer $container) {
        $this->photosRepository = $container->resolve("IPhotosRepository");
    }

    public function index() {
        ViewRenderer::render("Views/Home/Index.php", "Homepage");
    }

    public function about() {
        $viewModel = new AboutViewModel($this->photosRepository->getAll());
        ViewRenderer::render("Views/Home/About.php", "About me", $viewModel);
    }

    public function interests() {
        ViewRenderer::render("Views/Home/Interests.php", "My interests");
    }

    public function learning() {
        ViewRenderer::render("Views/Home/Learning.php", "Learning");
    }

    public function photos() {
        ViewRenderer::render("Views/Home/Photos.php", "Photo album");
    }

    public function history() {
        ViewRenderer::render("Views/Home/History.php", "History");
    }
}
\end{lstlisting}

ViewRenderer.php:
\begin{lstlisting}
class ViewRenderer
{
    static function render($viewName, $title, $viewModel = NULL, $layout = 'Layout.php')
    {
        include "Views/Shared/".$layout;
    }
}
\end{lstlisting}

Views/Shared/Layout.php:
\begin{lstlisting}
<!DOCTYPE html>

<html lang="en">
    <head>
        <title><?=$title?></title>
        <meta content="width=device-width" name="viewport" />
        <meta content="no-cache" http-equiv="Cache-Control" />
        <link href="../client-side/bundles/layout.bundle.css" rel="stylesheet" type="text/css" />
    </head>
    <body>
        <nav>
            <ul>
                <li><span>Web</span></li>
                <li><a href="/Home/Index">Home</a></li>
                <li><a href="/Home/About">About</a></li>
                <li><a href="/Home/Interests">My interests</a></li>
                <li><a href="/Home/Learning">Learning</a></li>
                <li><a href="/Home/Photos">Photos</a></li>
                <li><a href="/Contact/Index">Contact</a></li>
                <li><a href="/Test/Index">Test</a></li>
                <li><a href="/Home/History">History</a></li>
                <li><span id="time"></span><br/><span id="date"></span></li>
            </ul>
        </nav>
        <img src="../client-side/images/portrait.jpg" class="portrait" alt="Portrait picture in on the back. Don't mind it" />

        <div class="main-content">
            <?php include($viewName); ?>
        </div>

        <footer class="full-width">
            <section>
                <header><h2>Footer</h2></header>
                <p>Some useful links here</p>
                <ul>
                    <li>Link</li>
                    <li>Another link</li>
                </ul>
                <p>
                    And some info as well: Nunc eget viverra neque, non ullamcorper odio. Integer dolor sem, ultrices eu
                    tempus dapibus, sagittis ac lacus. Duis rutrum turpis eu lectus imperdiet, ac
                    sagittis orci faucibus. Vestibulum volutpat fringilla turpis vehicula lobortis.
                    Interdum et malesuada fames ac ante ipsum primis in faucibus. Aliquam posuere
                    tempor enim, et tempus tellus commodo mattis. Sed sollicitudin mauris vitae
                    vestibulum malesuada.
                </p>
            </section>
        </footer>
        <script src="../client-side/bundles/layout.bundle.js" >
    </body>
</html>
\end{lstlisting}

Views/Home/Index.php
\begin{lstlisting}
<link href="../client-side/bundles/home/index.bundle.css" rel="stylesheet" type="text/css" />

<header><h1>Portfolio</h1></header>
<section>
    <header><h2>About me</h2></header>
    <p><i>Gorbenko Kirill Nikolaevich.</i></p>
</section>
<section>
    <header><h2>NEXT CHAPTER</h2></header>
</section>

<script src="../client-side/bundles/home/index.bundle.js"></script>
\end{lstlisting}

Views/Home/About.php:
\begin{lstlisting}
<link href="../client-side/bundles/home/about.bundle.css" rel="stylesheet" type="text/css" />

<header>
    <h1>Autobiography</h1>
</header>
<section>
    <header>
        <h1>Section 1</h1>
    </header>
    <p>There should be some authobiography</p>
    <ol>
        <li>19 years old</li>
        <li>My parents's son.</li>
        <li>Dunno</li>
    </ol>
    <p>Photos:</p>
    <div class="photo-wrapper full-width">
        <?php
            foreach ($viewModel->photos as $photoName => $path) {
                echo("<img src='$path' alt='$photoName' />");
            }
        ?>
    </div>
</section>

<script src="../client-side/bundles/home/about.bundle.js"></script>
\end{lstlisting}

Views/Home/Interests.php:
\begin{lstlisting}
<link href="../client-side/bundles/home/interests.bundle.css" rel="stylesheet" type="text/css" />

<header>
    <h1>My interests</h1>
</header>
<section>
    <header>
        <h2>Contents</h2>
    </header>
    <button class="modal-button" id="dialog">Modal</button>
    <ul id="interests">
        <li class="contents" id="Hobbies">My hobbies</li>
        <li class="interests" id="Hobbies-expand">
            <h2>My hobbies</h2>
            <p>My hobbies are supposed to be here</p>
        </li>
        <li class="contents" id="Books">My books</li>
        <li class="interests" id="Books-expand">
            <h2>My books</h2>
            <p>My books are supposed to be here</p>
        </li>
        <li class="contents" id="Music">My music</li>
        <li class="interests" id="Music-expand">
            <h2>My music</h2>
            <p>My music is supposed to be here</p>
        </li>
        <li class="contents" id="Films">My films</li>
        <li class="interests" id="Films-expand">
            <h2>My films</h2>
            <p>My films are supposed to be here</p>
        </li>
    </ul>
</section>

<div class="modal">
    <div class="modal-window">
        <div>
            <p class="modal-message"></p>
        </div>
        <div class="modal-navigation">
            <button class="modal-button modal-confirm-button"></button>
            <button class="modal-button modal-cancel-button"></button>
        </div>
    </div>
</div>

<script src="../client-side/bundles/home/interests.bundle.js"></script>
\end{lstlisting}

Views/Home/Learning.php:
\begin{lstlisting}
<link href="../client-side/bundles/home/learning.bundle.css" rel="stylesheet" type="text/css" />

<section>
    <header>
        <h1>Disciplines</h1>
    </header>
    <h2>Институт информационных технологий и управления в технических системах</h2>
    <h2>Информационные системы и технологии</h2>
    <table>
        <tr>
            <td colspan="9">Disciplines by semesters</td>
        </tr>
        <tr>
            <th scope="row">First</th>
            <td>Math</td>
            <td>AIP</td>
            <td>Info</td>
            <td>Hitory</td>
            <td>Physycs</td>
            <td>Ecology</td>
            <td>Descrete Math</td>
            <td>English</td>
        </tr>
        <tr>
            <th scope="row">Second</th>
            <td>Math</td>
            <td>AIP</td>
            <td>Physics</td>
            <td>Philosophy</td>
            <td>Rights</td>
            <td>Economics</td>
            <td>BZD</td>
            <td>English</td>
        </tr>
        <tr>
            <th scope="row">Third</th>
            <td>Architecture</td>
            <td>Math</td>
            <td>Discrete math</td>
            <td>OOP</td>
            <td>TEC</td>
            <td>Economics</td>
            <td>Graphics</td>
            <td>English</td>
        </tr>
        <tr>
            <th scope="row">Fourth</th>
            <td>Electronics</td>
            <td>Algorithms</td>
            <td>Probability</td>
            <td>English</td>
            <td>System analysis</td>
            <td>Databases</td>
            <td>TSPP</td>
            <td>OS</td>
        </tr>
    </table>
</section>

<script src="../client-side/bundles/home/learning.bundle.js"></script>
\end{lstlisting}

Views/Home/Photos.php:
\begin{lstlisting}
<link href="../client-side/bundles/home/photos.bundle.css" rel="stylesheet" type="text/css" />

<header>
    <h1>Photoalbum</h1>
</header>
<div class="photo-wrapper full-width"></div>

<div id="lightbox">
    <div class="lightbox-content">
        <div class="image-wrapper">
            <img src="" />
        </div>
        <div>
            <div class="lightbox-button lightbox-arrow-left">&lt;</div>
            <div class="lightbox-button lightbox-arrow-right">&gt;</div>
        </div>
    </div>
    <div class="lightbox-close">×</div>
</div>

<script src="../client-side/bundles/home/photos.bundle.js"></script>
\end{lstlisting}

Views/Home/History.php:
\begin{lstlisting}
<link href="../client-side/bundles/home/history.bundle.css" rel="stylesheet" type="text/css" />

<header>
    <h1>History</h1>
</header>
<section>
    <header>
        <h2>Session history</h2>
    </header>
    <table class="session-history">
        <tr>
            <th colspan="2">Session history</th>
        </tr>
        <tr>
            <td>Home page</td>
            <td class="home-page"></td>
        </tr>
        <tr>
            <td>About me</td>
            <td class="about-page"></td>
        </tr>
        <tr>
            <td>My interests</td>
            <td class="interests-page"></td>
        </tr>
        <tr>
            <td>Learning</td>
            <td class="learning-page"></td>
        </tr>
        <tr>
            <td>Photo</td>
            <td class="photos-page"></td>
        </tr>
        <tr>
            <td>Contact</td>
            <td class="contact-page"></td>
        </tr>
        <tr>
            <td>Test</td>
            <td class="test-page"></td>
        </tr>
        <tr>
            <td>History</td>
            <td class="history-page"></td>
        </tr>
    </table>
</section>
<section>
    <header>
        <h2>Day History</h2>
    </header>
    <table class="global-history">
        <tr>
            <th colspan="2">All-time history</th>
        </tr>
        <tr>
            <td>Home page</td>
            <td class="home-page"></td>
        </tr>
        <tr>
            <td>About me</td>
            <td class="about-page"></td>
        </tr>
        <tr>
            <td>My interests</td>
            <td class="interests-page"></td>
        </tr>
        <tr>
            <td>Learning</td>
            <td class="learning-page"></td>
        </tr>
        <tr>
            <td>Photo</td>
            <td class="photos-page"></td>
        </tr>
        <tr>
            <td>Contact</td>
            <td class="contact-page"></td>
        </tr>
        <tr>
            <td>Test</td>
            <td class="test-page"></td>
        </tr>
        <tr>
            <td>History</td>
            <td class="history-page"></td>
        </tr>
    </table>
</section>

<script src="../client-side/bundles/home/history.bundle.js"></script>
\end{lstlisting}

\subsection{Реализация валидации форм}
Создадим контроллер с экшеном, отрисовывающим форму.

Controllers/ContactController.php:
\begin{lstlisting}
class ContactController
{
    public function __construct(IContainer $container) {}

    public function index() {
        $viewModel = new ValidationViewModel(true, array());
        ViewRenderer::render("Views/Contact/Index.php", "Contact me", $viewModel);
    }

    public function contactPost() {
        $validator = new ContactRequestValidator($_POST);
        $validationResult = $validator->validate();

        if ($validationResult->isValid) {
            ViewRenderer::render("Views/Contact/Success.php", "Contact me");
            return;
        }

        $viewModel = new ValidationViewModel($validationResult->isValid, $validationResult->errors);
        ViewRenderer::render("Views/Contact/Index.php", "Contact me", $viewModel);
    }
}
\end{lstlisting}

Views/Contact/Index.php:
\begin{lstlisting}
<link href="../client-side/bundles/contact/index.bundle.css" rel="stylesheet" type="text/css" />

<section>
    <header>
        <h1>Mail Me!</h1>
    </header>
    <p>All fields are required! Hover over fields to see the additional requirements.</p>

    <?php ViewHelper::writeValidationErrorsList($viewModel->errors); ?>

    <form action="Contact" method="POST">
        <ol>
            <li>
                <label for="full-name-input">Full Name</label>
                <input class="field-long" id="full-name-input" name="Name" type="text" />
                <br/>
            </li>
            <li>
                <label for="email-input">Email</label>
                <input class="field-long" id="email-input" name="Email" type="email" />
                <br/>
            </li>
            <li>
                <label for="phone-input">Phone number</label>
                <input class="field-long" id="phone-input" name="Phone" type="text" />
                <br/>
            </li>
            <li>
                <label for="message-input">Your Message</label>
                <textarea class="field-long" id="message-input" name="Message"></textarea>
            </li>
            <li>
                <label for="datepicker-input">Date</label>
                <input class="field-long" id="datepicker-input" name="Date" type="text" />
                <br/>
            </li>
            <li>
                <input type="submit" value="Submit" />
                <br/>
            </li>
        </ol>
    </form>
</section>

<script src="../client-side/bundles/contact/index.bundle.js"></script>
\end{lstlisting}

Views/Contact/Success.php
\begin{lstlisting}
<link href="../client-side/bundles/contact/success.bundle.css" rel="stylesheet" type="text/css" />

<h1>Thank you for your appeal!</h1>

<h2>I will try to contact you back as soon as I can!</h2>

<script src="../client-side/bundles/contact/success.bundle.js"></script>
\end{lstlisting}

Реализуем валидацию объекта \code{\textdollar\_POST} с помощью библиотеки \code{Valitron}.

Validators/BaseRequestValidator.php:
\begin{lstlisting}
abstract class BaseRequestValidator {
    protected array $post;

    public function __construct($post) {
        if (empty($post)) {
            http_response_code(ResponseCodes::$badRequestStatusCode);
            die();
        }
        $this->post = $post;
    }
}
\end{lstlisting}

Validators/ContactRequestValidator.php:
\begin{lstlisting}
class ContactRequestValidator extends BaseRequestValidator {
    public function __construct($post) {
        parent::__construct($post);
    }

    public function validate() {
        $v = new Validator($this->post);
        $v->rule('required', ['Name', 'Email', 'Phone', 'Message', 'Date']);
        $v->rule('regex', 'Name', '/[A-Za-z]+ [A-Za-z]+ [A-Za-z]+/');
        $v->rule('email', 'Email');
        $v->rule('regex', 'Phone', '/^[+][7|3]\d{9,11}$/');
        $v->rule('dateFormat', 'Date', 'd.m.Y');

        $isValid = $v->validate();
        $errors = $v->errors();

        return new ValidationResult($isValid, $errors);
    }
}
\end{lstlisting}

Validators/ValidationResult.php:
\begin{lstlisting}
class ValidationResult {
    public bool $isValid;
    public array $errors;

    public function __construct(bool $isValid, array $errors) {
        $this->isValid = $isValid;
        $this->errors = $errors;
    }
}
\end{lstlisting}

Создадим контроллер, обрабатывающий тест:

Controllers/TestController.php:
\begin{lstlisting}
class TestController {
    public function __construct(IContainer $container) {}

    public function index() {
        $viewModel = new ValidationViewModel(true, array());
        ViewRenderer::render("Views/Test/Index.php", "Test", $viewModel);
    }

    public function testPost() {
        $validator = new TestRequestValidator($_POST);
        $validationResult = $validator->validate();

        if ($validationResult->isValid) {
            $testVerification = new TestRequestVerification($_POST);
            $viewModel = new SuccessfulTestViewModel($testVerification->verify());
            ViewRenderer::render("Views/Test/Success.php", "Test", $viewModel);
            return;
        }

        $viewModel = new ValidationViewModel($validationResult->isValid, $validationResult->errors);
        ViewRenderer::render("Views/Test/Index.php", "Test", $viewModel);
    }
}
\end{lstlisting}

Validators/TestRequestValidator.php:
\begin{lstlisting}
class TestRequestValidator extends BaseRequestValidator {
    public function __construct($post) {
        parent::__construct($post);
    }

    public function validate() {
        $v = new Validator($this->post);
        $v->rule('required', ['Question1', 'Question2', 'Question3', 'Name', 'Email' ]);
        $v->rule('integer', 'Question1')->rule('in', 'Question1', [ 1, 2 ]);
        $v->rule('integer', 'Question2')->rule('in', 'Question2', [ 1, 2, 3 ]);
        $v->rule('regex', 'Name', '/[A-Za-z]+ [A-Za-z]+ [A-Za-z]+/');
        $v->rule('email', 'Email');

        $isValid = $v->validate();
        $errors = $v->errors();

        return new ValidationResult($isValid, $errors);
    }
}
\end{lstlisting}

Validators/TestRequestVerification.php:
\begin{lstlisting}
class TestRequestVerification {
    private array $post;

    public function __construct($post) {
        if (empty($post)) {
            throw new InvalidArgumentException("$post");
        }
        $this->post = $post;
    }

    public function verify() {
        $correctAnswers = array(
            'Question1' => '1',
            'Question2' => '2',
            'Question3' => 'Correct answer'
        );

        $areAnswersCorrect = true;
        foreach ($correctAnswers as $question => $answer) {
            if ($this->post[$question] != $answer) {
                $areAnswersCorrect = false;
            }
        }

        return $areAnswersCorrect;
    }
}
\end{lstlisting}

\section*{Выводы}
В ходе лабораторной работы было выполнено создание серверной части сайта на PHP
с использованием архитектуры MVC. Для создания удобочитаемых URL использован
\code{UrlRewrite} модуль PHP сервера. Для маршрутизации создан роутер. Валидация
параметров формы осуществлется с помощью библиотеки \code{Valitron}.

\end{document}