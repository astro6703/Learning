\documentclass[a4paper,14pt]{extarticle}

\usepackage{ucs}                                                                                                                   
\usepackage[utf8x]{inputenc}                                                                                                       
\usepackage[english,russian]{babel}
\usepackage[T2A]{fontenc}
\usepackage{extsizes}
\usepackage{tempora}
\usepackage[left=25mm, top=20mm, right=10mm, bottom=20mm, headheight=5pt]{geometry}
\usepackage{fancyhdr}
\usepackage{titling}
\usepackage{titlesec}
\usepackage{textcase}
\usepackage{indentfirst}
\usepackage{graphicx}
\usepackage{float}
\usepackage[labelsep=endash]{caption}
\usepackage{listings}
\usepackage{color}
\usepackage{enumitem}
\usepackage[font=normalsize]{subfig}
\usepackage{csquotes}
\usepackage{amsmath}

\graphicspath{ {./images/} }

\newcommand{\mylabnumber}{1}
\newcommand{\mylabtitle}{Исследование способов анализа областей эквивалентности
    и построения тестовых последовательностей}
\newcommand{\mysubject}{Тестиование программного обеспечения}
\newcommand{\mylecturer}{Тлуховская Н.П.}

\renewcommand{\baselinestretch}{1.25} % Sets basic line stretch
\renewcommand{\headrulewidth}{0pt} % Remove horizontal line below header in fancyhdr

\addto\captionsrussian{
    \renewcommand{\figurename}{Рисунок} % Set a default picture caption
    \renewcommand{\tablename}{Таблица} % Set a default table caption
}

\captionsetup[table]{singlelinecheck=false} % To make a table caption appear left-aligned

\pagestyle{fancy}
\lhead{} \rhead{} \cfoot{} % Setting empty headers
\chead{\thepage} % Sets central header page numbering

\setlength{\parindent}{1.25cm}
\setlength{\parskip}{8pt}

 % Format section style and indentations
\titleformat{\section}[hang]{\large \centering \bfseries}{\thesection}{0.5em}{\MakeTextUppercase}
\titlespacing{\section}{\parindent}{1em}{0em}

% Format subsection style and indentations
\titleformat{\subsection}[hang]{\bfseries}{\thesubsection}{0.5em}{}
\titlespacing{\subsection}{\parindent}{1em}{0em}

% Format subsubsection style and indentations
\titleformat{\subsubsection}[hang]{\normalfont}{\thesubsubsection}{0.5em}{}
\titlespacing{\subsubsection}{\parindent}{1em}{0em}

% Format enumerate style with "enumitem" package
\setlist[enumerate, 1]{wide=\parindent,leftmargin=0pt,topsep=0pt,
    itemsep=0pt,partopsep=0pt,parsep=0pt}
\setlist[enumerate, 2]{wide=2\parindent,label=\alph*.,leftmargin=1.25cm,topsep=0pt,
    itemsep=0pt,partopsep=0pt,parsep=0pt}
\setlist[enumerate, 3]{wide=3\parindent,label=\alph*.,leftmargin=2.5cm,topsep=0pt,
    itemsep=0pt,partopsep=0pt,parsep=0pt}

% Format itemize style with "enumitem" package
\setlist[itemize, 1]{wide=\parindent,leftmargin=0pt,topsep=0pt,itemsep=0pt,
    partopsep=0pt,parsep=0pt}
\setlist[itemize, 2]{wide=2\parindent,leftmargin=1.25cm,topsep=0pt,itemsep=0pt,
    partopsep=0pt,parsep=0pt}
\setlist[itemize, 3]{wide=3\parindent,leftmargin=2.5cm,topsep=0pt,itemsep=0pt,
    partopsep=0pt,parsep=0pt}

\captionsetup[figure]{justification=centering}

\begin{document}

    \lstset{ % "listings package configuration"
        basicstyle=\footnotesize\ttfamily,
        breaklines=true,
        numbersep=5pt,
        tabsize=4,
        gobble=8,
        extendedchars=\true,
        keepspaces=\true,
        numbers=left,
        stringstyle=\ttfamily,
        showstringspaces=\false
    }

    % ############################################################################
    % -------------------------------- Title page --------------------------------
    % ############################################################################

    \begin{titlepage}
        
        \thispagestyle{empty}
        
        \begin{center}
            
            Министерство науки и высшего образования Российской Федерации \\
            Севастопольский государственный университет \\
            Кафедра ИС
            
            \vfill

            Отчет \\
            по лабораторной работе №\mylabnumber \\
            \enquote{\mylabtitle} \\
            по дисциплине \\
            \enquote{\MakeTextUppercase{\mysubject}}

        \end{center}

        \vspace{1cm}

        \noindent\hspace{7.5cm} Выполнил студент группы ИС/б-17-2-о \\
        \null\hspace{7.5cm} Горбенко К. Н. \\
        \null\hspace{7.5cm} Проверила \\
        \null\hspace{7.5cm} \mylecturer

        \vfill

        \begin{center}
            Севастополь \\
            2019
        \end{center}

    \end{titlepage}

    % ############################################################################
    % ------------------------------ Document start ------------------------------
    % ############################################################################

    \section{Цель работы}

    Исследовать способы анализа областей эквивалентности входных данных 
    для тестирования программного обеспечения. Приобрести практические
    навыки составления и построения тестовых последовательностей.

    \section{Задание на работу}

    Для \textbf{варианта № 5} заданы следующие требования к программам: 

    \begin{enumerate}
        \item Дана целочисленная квадратная матрица. Определить сумму элементов
              в тех столбцах, которые не содержат отрицательных элементов.
        \item Дана строка. Удалить в данной строке символ, стоящий на заданной
              позиции.
        \item Программа, которая считывает текст из файла и выводит его на
              экран, меняя местами каждые два соседних слова.
    \end{enumerate}

    Для каждой из программ необходимо:

    \begin{enumerate}
        \item Написать программу, выполняющую заданные действия.
        \item Определить области эквивалентности входных данных.
        \item Составить примеры тестовых последовательностей.
    \end{enumerate}

    \section{Текст программных модулей}
    \subsection{Операции над матрицами}

    Была написана следующая программа:

    \begin{lstlisting}[language={[Sharp]C}]
        public static class MatrixOperations
        {
            public static IEnumerable<int> GetSumOfColumnsWithoutNegativeElements(int[,] matrix)
            {
                if (matrix == null) throw new ArgumentNullException($"{nameof(matrix)} instance was null");
                if (matrix.GetLength(0) != matrix.GetLength(1)) throw new ArgumentException($"Only square martixes are allowed");

                return matrix
                        .GetColumns()
                        .Where(column => column.All(items => items >= 0))
                        .Select(column => column.Sum());
            }
        }

        public static IEnumerable<int[]> GetColumns(this int[,] array)
        {
            if (array == null) throw new ArgumentNullException($"{nameof(array)} instance was null");

            for (var j = 0; j < array.GetLength(1); j++)
            {
                var column = new int[array.GetLength(0)];

                for (var i = 0; i < array.GetLength(0); i++)
                    column[i] = array[i, j];

                yield return column;
            }
        }
    \end{lstlisting}

    \subsection{Операции над строками}

    Была написана следующая программа:

    \begin{lstlisting}[language={[Sharp]C}]
        public static class StringOperations
        {
            public static string RemoveAt(string source, int position) => source.Remove(position, 1);
        }
    \end{lstlisting}

    \subsection{Операции над текстом}

    \begin{lstlisting}[language={[Sharp]C}]
        public class TextOperations
        {
            private readonly StreamReader streamReader;
            private readonly string[] delimiters = { " ", ".", ",", "?", "!", "(", ")", ":", ";", Environment.NewLine };

            public TextOperations(StreamReader streamReader)
                => this.streamReader = streamReader ?? throw new ArgumentNullException($"{nameof(streamReader)} instance was null");

            public string ReverseEveryTwoWords()
            {
                var source = streamReader.ReadToEnd();

                return string.Join(" ", source
                                            .Split(delimiters, StringSplitOptions.RemoveEmptyEntries)
                                            .Select((word, i) => new { Value = word, Index = i })
                                            .GroupBy(x => x.Index / 2)
                                            .Select(group => group.Select(x => x.Value).Reverse())
                                            .SelectMany(x => x)
                    );
            }
        }
    \end{lstlisting}

    \section{Составление областей эквивалентности и тестовых примеров}
    \subsection{Тестирование программы № 1}

    Определим области эквивалентности.
    \begin{enumerate}
        \item По размеру матрицы:
        \begin{enumerate}
            \item Матрица состоит из одного элемента.
            \item Матрица состоит из нескольких элементов.
        \end{enumerate}
        \item По наличию и расположению отрицательных элементов:
        \begin{enumerate}
            \item Отрицательные элементы отсутствуют.
            \item Существует один отрицательный элемент.
            \item Существует несколько отрицательных элементов.
        \end{enumerate}
    \end{enumerate}

    Составим тестовые примеры:

    1.a,2.a: [3];

    1.a,2.b: [-3];

    1.b,2.a: $$\begin{pmatrix} 15, 10, 36 \\ 23, 0, 14 \\ 62, 15, 35 \end{pmatrix}$$

    1.b,2.b: $$\begin{pmatrix} 13, 2 , 14, 16 \\ 0, 54, -2, 9 \\ 62, 0, 21, 15 \\ 64, 0, 11, 2 \end{pmatrix}$$
    
    1.b,2.c: $$\begin{pmatrix} 85, 76 , -53, 16,  \\ 0, 637, -2, -12 \\ -15, 0, 5, 90 \\ 64, 126, 11, 2 \end{pmatrix}$$

    \subsection{Тестирование программы № 2}

    Опредедим области эквивалентности:
    \begin{enumerate}
        \item По размеру строки:
        \begin{enumerate}
            \item Пустая строка.
            \item Длина строки - 1.
            \item Длина строки > 1.
        \end{enumerate}
        \item По позиции удаляемого символа:
        \begin{enumerate}
            \item Удаляется символ за пределами множества индексов строки.
            \item Удаляется первый элемент строки.
            \item Удаляется последний элемент строки.
            \item удаляется средний элемент строки.
        \end{enumerate}
    \end{enumerate}

    Составим тестовые примеры:

    1.a: \enquote{}, любой индекс.
    
    1.b,2.a: \enquote{a}, 1.
    
    1.b,2.b: \enquote{b}, 0.

    1.b,2.c: \enquote{C}, 0.

    1.c,2.a: \enquote{The quick brown fox jumps over the lazy dog.}, 44.
    
    1.c,2.b: \enquote{documentclass[a4paper,14pt]{extarticle}}, 0.

    1.c,2.c: \enquote{Hello, here is some text without a meaning.  This text should show what a printed text will look like at this place.  If you read this text, you will get no information.}, 173.

    1.c,2.d: \enquote{(var i = 0; i < array.GetLength(0); i++}, 14.

    \subsection{Тестирование программы № 3}

    Опредедим области эквивалентности:
    \begin{enumerate}
        \item По количеству слов:
        \begin{enumerate}
            \item Пустая строка
            \item Одно слово
            \item Четное количество слов
            \item Нечетное количество слов
        \end{enumerate}
    \end{enumerate}

    Создадим тестовые примеры:

    1.a: \enquote{}.

    1.b: \enquote{Высокопревосходительство}.

    1.c: \enquote{После того, как определены области эквивалентности, выбираются
    тестовые последовательности, принадлежащие каждой из областей. При этом
    руководствуются следующими правилами}.

    1.d: \enquote{ветвей это метод структурного тестирования, при котором
    проверяются все независимо выполняемые ветви компонента или программы.
    Если выполняются все независимые ветви, то и все операторы должны
    выполняться, по крайней мере}.

    \section{Выводы}

    В ходе лабораторной работы было исследовано тестирование ПО,
    представляющего собой черный ящик. Для тестирования черного ящика
    необходимо разбить множество возможных входных данных программы на
    области эквивалентности, на которых, как предполагается, программа ведет
    себя одинаково. После этого создаются тестовые примеры, содержащие входные
    данные из разных областей. 
    
    При тестировании черного ящика крайне необходимо создавать тестовые примеры,
    содержащие различные комбинации разных видов областей входных данных и их размера.
\end{document}