\documentclass[a4paper,14pt]{extarticle}
\usepackage{../../tex-shared/report-layout}

\renewcommand{\mylabnumber}{1}
\renewcommand{\mylabtitle}{Исследование в области устава проекта}
\renewcommand{\mysubject}{Управление IT-проектами}
\renewcommand{\mylecturer}{Смирнова Н.Б.}

\begin{document}
\begin{titlepage}
    
    \thispagestyle{empty}
    
    \begin{center}
        
        Министерство науки и Высшего образования Российской Федерации \\
        Севастопольский государственный университет \\
        Кафедра ИС
        
        \vfill

        Отчет \\
        по лабораторной работе №\mylabnumber \\
        \enquote{\mylabtitle} \\
        по дисциплине \\
        \enquote{\MakeTextUppercase{\mysubject}}

    \end{center}

    \vspace{1cm}

    \noindent\hspace{7.5cm} Выполнил студент группы ИС/б-17-2-о \\
    \null\hspace{7.5cm} Горбенко К. Н. \\
    \null\hspace{7.5cm} Проверил \\
    \null\hspace{7.5cm} \mylecturer

    \vfill

    \begin{center}
        Севастополь \\
        \the\year{}
    \end{center}

\end{titlepage}

\section{Цель работы}
Исследовать способы составления устава проекта.

\section{Задание на работу}
Выбрать проект (техническое задание/задание на одну из лабораторных или курсовых
работ) и составить устав проекта по шаблону.

\section{Ход работы -- составлене устава проекта}
\subsection{Общая информация о проекте}

\begin{table}[H]
    \caption{Общая информация о проекте}
    \begin{tabular}{ | p{5.5cm} | p{11cm} | }
        \hline
        Наименование проекта & Разработка адаптивного сервиса для изучения иностранной лексики \\ \hline
        Краткое наименование проекта & Разработка адаптивного сервиса для изучения иностранной лексики \\ \hline
        Заказчик проекта & Севастопольский государственный университет \\ \hline
        Дата начала проекта & 01.03.2021 \\ \hline
        Дата окончания проекта & 31.06.2021 \\ \hline
    \end{tabular}
\end{table}

\subsection{Описание проекта}
\subsubsection{Предпосылки проекта}
Дипломный проект на тему \enquote{Адаптивная система в помощь изучающему
иностранный язык} зародился из желания преодолеть языковой барьер при
общении/потреблении контента на иностранном языке. При повышении уровня владения
языком в грамматическом плане всегда возникает необходимость для дальнейшего
развития обращаться к оригинальным источникам. В этой ситуации всегда
проявляется нехватка словарного запаса изучаемого языка, которая приводит
учащегося к постоянному обращению к словарям. Проблемой, которую данный проект
призван решить, является необходимость постоянно в каком-либо виде записывать
недавно изученную лексику (иначе она очень быстро забывается и при повторной
встрече ранее изученной лексики велики шансы быть вынужденным обращаться к
словарям снова). Инструменты, решающие эту проблему уже существуют, но они, в
основном, делятся на два разных типа, решающих разные задачи. Задачей этого
проекта является попытка объединения таких типов приложений.

\subsubsection{Обонование целесообразности проведения проекта}
Существующие сервисы решают при изучении лексики две разные задачи: запоминание
(с помощью различных упражнений, периодических напоминаний, возможности
восприятия на слух и т.д.) и предоставление всевозможной контекстной к лексике
информации (возможности использования, примеры предложений и т.д.). Если
пытаться обе задачи выполнить одновременно, придется использовать различные
сервисы (и поддерживать две коллекции несинхронизированной лексики). Таким
образом, существует необходимость в создании сервиса, объединившего бы два этих
подхода.

\subsubsection{Объем проекта}
Объем проекта включает работы по проектированию, реализации и внедрению
логически завершенных и взаимосвязанных функциональных блоков и компонентов
продукта проекта и определяется в трех ракурсах:

\begin{itemize}
    \item функциональный объем определяет функциональные характеристики\\
          внедряемого продукта. Определяется техническим заданием, согласованным с
          руководителем проекта;
    \item организационный объем, который состоит в планировании проекта на
          начальных этапах;
    \item технический объем, который определяется требованием нормального\\
          функционирования продукта проекта в заданном функциональном и
          организационном объеме;
\end{itemize}

\subsubsection{Продукт проекта}
Продуктом проекта является информационная система, представляющая собой
адаптивную систему для ассистирования изучающему английский язык.

\subsubsection{Заинтересованные в развитии проекта стороны}
\begin{itemize}
    \item Заказчик проекта.
    \item Руководитель проекта.
\end{itemize}

\subsection{Цели и ожидаемые результаты}
Цели:
\begin{itemize}
    \item автоматизация процесса создания пользовательского словаря;
    \item большая свобода пользователя в редактировании или изменении
          предоставляемых сервисом переводов, контекстов и т.д. (персонализация);
    \item упрощение процесса повторения изученной лексики;
    \item повышение эффективности изучения лексики (больший объем запоминаемой
          информации на одну лексическую единицу);
\end{itemize}

Ожидаемые результаты:
\begin{itemize}
    \item время, затрачиваемое на создание в персональном словаре лексической
          единыцы в пределах 2 минут (при этом устранена необходимость вручную
          вводить переводы и контексты изучаемой лексики);
    \item пользователь получит удобный доступ к своим словарям с возможностью
          быстрого поиска и редактирования;
    \item пользователь получит возможность удобно проходить процесс повторения
          изученной лексики через механизм автоматического создания упражнений;
    \item пользователь получит возможность просмотреть статистику выполненных
          упражнений;
\end{itemize}

\subsection{Ограничения проекта}
\begin{table}[H]
    \caption{Ограничения проекта}
    \begin{tabular}{ | p{5.425cm} | p{11cm} | }
        \hline
        Название ограничения & Описание \\ \hline
        Время исполнения проекта & до 31.06.2021 \\ \hline
        Время планирования проекта & до 31.04.2021 \\ \hline
        Финансовые затраты по проекту & -- \\ \hline
        Организационные & Работы по планированию функциональности проекта (составлению технического задания) осуществляются во взаимодействии с руководителем проекта \\ \hline
    \end{tabular}
\end{table}

\subsection{Участники проекта}
Участники со стороны заказчика:
\begin{table}[H]
    \caption{Участники проекта со стороны заказчика}
    \begin{tabular}{ | p{4cm} | p{5cm} | p{7cm} | }
        \hline
        ФИО & Должность в организации & Роль в проекте \\ \hline
        Карлусов В.Ю. & Доцент кафедры ИС & Принятие решений по изменению типа функциональных задач информационной системы \\ \hline
        Строганов В.А. & Ст. преподаватель кафедры ИС & Консультация при принятии решений о реализации функциональности системы; проверка соответствия реализованной функциональности техническому заданию \\ \hline
    \end{tabular}
\end{table}

Участники со стороны исполнителя:
\begin{table}[H]
    \caption{Участники проекта со стороны исполнителя}
    \begin{tabular}{ | p{4cm} | p{5cm} | p{7cm} | }
        \hline
        ФИО & Должность в организации & Роль в проекте \\ \hline
        Горбенко К.Н. & Исполнитель & Разработка технического задания проекта;
        реализация функциональных требований (как на стороне сервера, так и на
        стороне клиента); общее тестирование системы; составление тест-кейсов;
        разработка общего дизайна интерфейса; компоновка дизайнерских решений;
        графический дизайн; \\ \hline
    \end{tabular}
\end{table}

\subsection{Критерии успешности реализации проекта}
Проект считается успешно завершенным при достижении следующих результатов:
\begin{itemize}
    \item система становится доступна пользователю;
    \item реализованы следующие функции системы: возможность автоматизированного
          добавления лексической единицы с использованием сторонных словарей и
          переводчиков, возможность получить и выполнить упражнения с уже
          добавленной лексикой;
    \item дата ввода системы в эксплуатацию соответствует установленным
          заказчиком срокам;
\end{itemize}

\subsection{Риски проекта}
Основными рисками проекта являются:
\begin{itemize}
    \item Поскольку исполнителем является один человек, существует риск, что
          проект останется без действующих исполнителей;
    \item Отсутствие доступных провайдеров хостинговых услуг для приложений и БД;
    \item Ошибки при планировании работ по проекту;
    \item Ошибки в организации работы по проекту;
    \item Отсутствие документирования проекта;
    \item Реализация несуществующей функциональности;
    \item Разработка неудобного пользовательского интерфейса;
    \item Недостатки в работе внешних словарей и переводчиков;
\end{itemize}

\subsection{Фазы жизненного цикла проекта}
\begin{table}[H]
    \caption{Фазы жизненного цикла проекта}
    \begin{tabular}{ | p{5cm} | p{3cm} | p{8cm} | }
        \hline
        Название фазы & Временные рамки & Описание \\ \hline
        Инициация и запук проекта & 01.03.2021 & решение о целесообразности проекта \\ \hline
        Предпроектное обследование и получение полного технического задания & c 01.03.2021 по 15.03.2021 & консультация с заказчиком, выяснение требований и составление технического задания \\ \hline
        Проектирование & с 15.03.2021 по 31.03.2021 & разработка технического проекта на основании согласованного технического задания \\ \hline
        Реализация проекта & с 01.04.2021 по 31.05.2021 & создание информационной системы согласно техническому проекту \\ \hline
        Опытно-промышленная эксплуатация & c 01.06.2021 по 31.06.2021 & тестирование системы; устранение обнаруженных ошибок; подготовка отчетных документов для сдачи проекта \\ \hline
        Завершение проекта & 31.06.2021 & переход в промышленный режим эксплуатации; поддержка кодовой базы проекта; подведение итогов проекта, оценка достижения целей и задач проекта; принятие решения о развитии ИС \\ \hline
    \end{tabular}
\end{table}

\section*{Выводы}
В ходе лабораторной работы был изучен состав устава проекта и составлен устав
для проекта \enquote{Разработка адаптивного сервиса для изучения иностранной
лексики}.

Устав проекта -- документ, содержащий общую информацию о проекте:
\begin{itemize}
    \item причины и цели создания проекта;
    \item обоснование проекта;
    \item структура управления проектом, распределение ролей и т.д.;
    \item календарный план проекта;
    \item признаки завершения проекта;
    \item риски и т.д.
\end{itemize}

Если воспринимать устав проекта как документа, обеспечивающего согласованность
действий все участников проекта, то многие части устава не имеют смысла в том
случае, когда проект исполняется одним человеком в его личных интересах.
\end{document}